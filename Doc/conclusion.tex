%Set the document class and font size
\documentclass[fleqn, bachelor,subf,12pt,notitlepage]{article}
\usepackage[utf8]{inputenc}
\usepackage{enumerate}
\usepackage{amsmath}
\usepackage{mathtools} 
\usepackage{amssymb}
\usepackage{systeme}
\usepackage[english]{babel}
\usepackage{xparse}
\usepackage{xfrac}
\usepackage{setspace}
\usepackage{multicol}
\usepackage{array}
\usepackage{tabularx}
\usepackage{bigints}
\usepackage{fontspec}
\usepackage{chngcntr}
\usepackage{caption}

%This package allows to modify enumerations
\usepackage{enumitem}


%This package allows to change figure insertion mode (H, etc.)
\usepackage{float}

%This package is used for big sum symbol
\usepackage{relsize}

\usepackage
[
 	a4paper,
 	left=30mm,
	right = 10mm,
 	top=20mm,
	bottom=25mm
 ]
{geometry}

%This command sets the font
\setmainfont{Sylfaen}

%These commands change line spacing
%\onehalfspacing
\linespread{1.5}

\title{Դիպլոմային աշխատանք}
\author{Կամո Սևոյան}

%This command allows to change page counter.
\setcounter{page}{93}

\begin{document}

\newpage
\section*{\centering {\addfontfeatures{FakeBold=2.0}Եզրակացություն}}
\sloppy

\hspace{\parindent}Այսպիսով, ավարտական աշխատանքում քննարկվել են մինչև երեք փոփոխականի ֆունկցիաների մոտարկման եղանակները, ավելի կոնկրետ՝ Լագրանժի և էրմիթյան մոտարկման եղանակները, հատկապես ուշադրություն դարձնելով երկու փոփոխականի ֆունկցիաներին։ Կառուցված մոտարկման բանաձևերի օգնությամբ իրականացվել են մասնակի ածանցյալներով դիֆերենցիալ հավասարումների մոտավոր լուծումներ վարիացիոն մեթոդով։ Իրականացման տեսանկյունից հատկապես բարդ է Արգիրիսի բազիսային ֆունկցիաները, քանի որ դեպի ստանդարտ էլեմենտ և հակադարձ ձևափոխությունները բավականին բարդ են (Պիոլայի ձևափոխություն) ի համեմատ մյուս բազիսային էլեմենտների։ 

Հաշվարկներն իրականացնելու համար օգտագործվել է Python ծրագրավորման լեզուն Jupyter Lab միջավայրում, որը թույլ է տալիս կոդի ինտերակտիվ կատարում։ Սիմվոլիկ հաշվարկները հեշտացնելու նպատակով օգտագործվել է Sympy գրադարանը, որը թույլ է տալիս իրականացնել սիմվոլիկ դիֆերենցում և ինտեգրում։ Եռանկյունացում կատարելու համար օգտագործվել է Triangle գրադարանը։ Թվային հաշվարկներ կատարելու համար օգտագործվել է Numpy գրադարանը, որը հարմար է բազմաչափ զանգվածների հետ տարատեսակ գործողություններ կատարելու համար։

Որպես ամփոփում, աշխատանքի հետագա շարունակության համար կարելի է դիտարկել վեկտորական և մատրիցային ֆունկցիաների մոտարկման խնդիրը։



\newpage
\section*{\centering {\addfontfeatures{FakeBold=2.0}Գրականության ցանկ}}

%\setlist[enumerate]{itemindent=\dimexpr\labelwidth+\labelsep\relax,leftmargin=0pt}
\begin{enumerate}[leftmargin=0.5cm]
\item {\addfontfeatures{FakeBold=2.0}A. R. Mitchell}, The finite element method in partial differential equations, London, Wiley, 1977.
\item {{\addfontfeatures{FakeBold=2.0}R. Courant and D. Hilbert},} Methods of mathematical physics, Berlin, Verlag von Julius Springer, 1924.
\item{\addfontfeatures{FakeBold=2.0}Robert C. Kirby}, A general approach to transforming finite elements, Amsterdam, Elsevier, 2018.
\item{\addfontfeatures{FakeBold=2.0}Р.З. Даутов, М.М. Карчевский}, Введение в теорию метода конечных элементов, Казань, Казанский государственный университет
им. В.И. Ульянова–Ленина, 2004.
\item{\addfontfeatures{FakeBold=2.0}А.Н.Тихонов, А.А.Самарский}, Уравнения математической физики, Государственное издательство технико-теоретической литературы, Москва 1951.
\item{\addfontfeatures{FakeBold=2.0}JupyterLab: A Next-Generation Notebook Interface}, https://jupyter.org.
\item{\addfontfeatures{FakeBold=2.0}Numeric Python}, https://numpy.org.
\item{\addfontfeatures{FakeBold=2.0}Symbolic Python}, https://sympy.org.
\item{\addfontfeatures{FakeBold=2.0}A Two-Dimensional Quality Mesh Generator and Delaunay Triangulator}, http://cs.cmu.edu.
\item{\addfontfeatures{FakeBold=2.0}An encyclopedia of finite element definitions}, https://defelement.com.
\item{\addfontfeatures{FakeBold=2.0}Խալաթյան Ռ.Պ., Սարգսյան Ս.Հ}, Կենսագործունեության անվտանգություն։

\end{enumerate}

\end{document}
