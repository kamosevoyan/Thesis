%Set the document class and font size
\documentclass[fleqn, bachelor,subf,12pt,notitlepage]{article}
\usepackage[utf8]{inputenc}
\usepackage{enumerate}
\usepackage{amsmath}
\usepackage{mathtools} 
\usepackage{amssymb}
\usepackage{systeme}
\usepackage[english]{babel}
\usepackage{xparse}
\usepackage{xfrac}
\usepackage{setspace}
\usepackage{multicol}
\usepackage{array}
\usepackage{tabularx}
\usepackage{bigints}
\usepackage{fontspec}
\usepackage{chngcntr}
\usepackage{caption}

%This package allows to modify enumerations
\usepackage{enumitem}


%This package allows to change figure insertion mode (H, etc.)
\usepackage{float}

%This package is used for big sum symbol
\usepackage{relsize}


%This command puts section number after section name
%\usepackage[explicit]{titlesec}
%\titleformat{\section}{\normalfont\Large\bfseries}{}{0em}{#1\ \thesection}

%This command allows to add section number in figure caption
%\counterwithin{figure}{section}

%This command help to keep counting but allows not to print the number of the section, etc.
%\renewcommand\thesection{}
%\renewcommand\thesubsection{}

%Change command to armenian
%\addto\captionsenglish{\renewcommand{\figurename}{Նկար}}

%This is a command for creating new command for reducing the size of the given math equation
\newcommand\scalemath[2]{\scalebox{#1}{\mbox{\ensuremath{\displaystyle #2}}}}

\usepackage
[
 	a4paper,
 	left=30mm,
	right = 10mm,
 	top=20mm,
	bottom=25mm
 ]
{geometry}

%This command sets the font
\setmainfont{Sylfaen}

%These commands change line spacing
%\onehalfspacing
\linespread{1.5}

\title{Դիպլոմային աշխատանք}
\author{Կամո Սևոյան}

%This command allows to change page counter.
\setcounter{page}{6}

\begin{document}

\section*{\centering {\addfontfeatures{FakeBold=2.0}Ներածություն}}

%This command allows to correctly
\sloppy

\hspace{\parindent}Մասնակի ածանցյալներով դիֆերենցիալ հավասարումները հիմնարար դեր են կատարում տարբեր ֆիզիկական երևույթների նկարագրման համար, ինչպիսիք են ջերմափոխանակությունը, հեղուկների դինամիկան, պինդ մարմինների դեֆորմացիա և այլն։ Քանի որ ընդհանուր դեպքում դրանց անալիտիկ լուծում ստանալը հնարավոր չէ, մշակվել են մոտավոր մեթոդներ դրանց լուծման մոտարկման համար։ 

Այդ մեթոդների մեծ մասի հիմք է հանդիսանում բազմաչափ մոտարկումը, որը թույլ է տալիս մոտարկել հավասարման լուծումը վերջավոր քանակի կետերի միջոցով։
Մոտարկման ճշգրտությունը կախված է օգտագործվող մեթոդից և հանգուցային կետերի քանակից ու դրանց դասավորվածությունից։ Ի տարբերություն մեկ փոփոխականի ֆունկցիայի,  երկու և ավել փոփոխականի ֆունկցիաների դեպքում խնդիրն առավել բարդ է, հատկապես հանգուցային կետերի ոչ կարգավորված դասավորվածության դեպքում։

Վարիացիոն հաշվի միջոցով կարելի է էապես հեշտացնել խնդրի լուծումը, քանի որ այն թույլ է տալիս հավասարման լուծումը մոտարկել պակաս դիֆերենցելիության կարգ ունեցող ֆունկցիաներով։ Այս դեպքում դիֆերենցիալ հավասարման լուծման խնդիրը բերվում է որևէ ինտեգրալ ֆունկցիոնալի մինիմումի որոնման, որտեղ կիրառելով մոտարկման բանաձևերը, ելակետային խնդրի լուծումը վեր է ածվում գծային հավասարումերի համականգի լուծման փնտրման։

Աշխատանքում քննարկվելու են միաչափ, երկչափ և եռաչափ մոտարկման տարբեր եղանակեր, որոնք կիրառվելու են տարբեր կարգի դիֆերենցիալ հավասարումների լուծումները մոտարկելու համար։
\newpage

%This command builds content automatically
%	\tableofcontents

\section*{\centering {\addfontfeatures{FakeBold=2.0}Գլուխ 1\\ Միաչափ մոտարկում}}
\subsection*{{\addfontfeatures{FakeBold=2.0}1.1 Էրմիթյան մոտարկում}}
\hspace{\parindent}Նախքան անդրադառնալը բազմաչափ մոտարկման խնդրին, քննարկենք մեկ փոփոխականի ֆունկցիայի մոտարկման որոշ մանրամասներ։


Ենթադրենք տրված են $f:\Omega \mapsto \Theta, \Omega, \Theta \subset \mathbb{R}$ ֆունկցիան,  $\left\{x_{i}\right\}_{i=0}^{N}$ կետերը և դրանց համապատասխան $\left\{y_{i}=f\left(x_{i}\right)\right\}_{i=0}^{N}$ արժեքները։ Յուրաքանչյուր $\left[x_{i}, x_{i+1}\right]$ հատվածում  մոտարկող ֆունկցիան իրենից ներկայացնում է գծային ֆունկցիա, որը կարելի է ներկայացնել հետևյալ տեսքով.
\begin{equation}
p_{1}^{(i)}\left(x\right)=\dfrac{x_{i+1}-x}{x_{i+1}-x_{i}}y_{i}+\dfrac{x-x_{i+1}}{x_{i+1}-x_{i}}y_{i+1}, i=\overline{0, N-1}
\end{equation}
Այսպիսով $\left[x_{0}, x_{N}\right]$ հատվածում կտոր առ կտոր մոտարկող ֆունկցիան տրվում է հետևյալ կերպ.
\begin{equation}
p_{1}\left(x\right)=\sum_{i=0}^{N}\varphi_{i} \left(x\right)y_{i}
\end{equation}
Որտեղ 
\begin{equation}
\begin{aligned}
\varphi_{0}\left(x\right)&=\begin{cases}
\dfrac{x_{1}-x}{x_{1}-x_{0}}, x\in \left[x_{0}, x_{1}\right]\\
0, x\in \left[x_{1}, x_{N}\right]\\
\end{cases}\\
\varphi_{i}\left(x\right)&=\begin{cases}
0, x\in \left[x_{0}, x_{i-1}\right]\\
\dfrac{x-x_{i-1}}{x_{i}-x_{i-1}}, x\in \left[x_{i-1}, x_{i}\right]\\
\dfrac{x_{i+1}-x}{x_{i+1}-x_{i}}, x\in \left[x_{i}, x_{i+1}\right]\\
0, x\in \left[x_{i+1}, x_{N}\right]\\
\end{cases}\\
\varphi_{N}\left(x\right)&=\begin{cases}
0, x\in \left[x_{0}, x_{N-1}\right]\\
\dfrac{x-x_{N-1}}{x_{N}-x_{N-1}}, x\in \left[x_{N-1}, x_{N}\right]\\
\end{cases}
\end{aligned}
\end{equation}
$\varphi_{i}\left(x\right)$ ֆունկցիաները կոչվում են բազիսային ֆունկցիաներ, որոնք ունեն այսպես կոչված լոկալ կրիչներ, քանի որ դրանք ոչ զրոյական են որևէ տիրույթում և զրոյական որոշման տիրույթի մնացած մասերում։
Նմանատիպ բազիսային ֆունկցիաների հիմնական հատկությունն այն է, որ դրանք հավասար են մեկի որևէ կոնկրետ հանգույցում և հավասար են զրոյի մնացած բոլոր հանգույցներում։ Նշենք սակայն, որ այս տիպի մոտարկումը $C^{0}$ դասի է, այսինք միայն անընդհատ է, և հետևաբար կիրառելի չէ այն խնդիրներում, որտեղ պահանջվում է ավելի բարձր կարգի ողորկություն։
\begin{figure}[H]
\centering
\includegraphics[width=1.0\textwidth]{images/one_var_linear}
\captionsetup{labelformat=empty}
\caption{Նկար 1.1. Գծային մոտարկման բազիսային ֆունկցիաներ։}
\end{figure}
Այժմ դիտարկենք հետևյալ խնդիրը. անհրաժեշտ է կառուցել  կտոր առ կտոր մոտարկող ֆունկցիա, որը ֆունկցիայի արժեքի հետ մեկտեղ կհամընկնի նաև ֆունկցիայի առաջին կարգի ածանցյալի արժեքի հետ մոտարկման հանգույցներում։ Այսինքն.
\begin{equation}
\dfrac{d^{j}}{dx^j}f(x_{i})=\dfrac{d^{j}}{dx^j}p_{3}(x_{i}),   \; j=0, 1;  \; i=\overline{0, N-1}
\end{equation}
Յուրաքանչյուր $\left[x_{i}, x_{i+1}\right]$ հատվածում մոտարկող ֆունկցիան իրենից ներկայացնում է խորանարդային ֆունկցիա, որը կարելի է ներկայացնել հետևյալ տեսքով։
\begin{equation}
p_{3}^{\left(i\right)} = \alpha_{i}(x)f(x_{i})+\beta_{i+1}(x)f(x_{i+1})+\gamma_{i}(x)f^{'}(x_{i})+\delta_{i+1}(x)f^{'}(x_{i+1})
\end{equation}
որտեղ 
\begin{equation}
\begin{aligned}
&\alpha_{i}(x)=\dfrac{\left(x_{i+1}-x\right)^{2}\left[\left(x_{i+1}-x_{i}\right)+2\left(x-x_{i}\right)\right]}{\left(x_{i+1}-x_{i}\right)^{3}}\\
&\beta_{i+1}(x)=\dfrac{\left(x-x_{i}\right)^{2}\left[\left(x_{i+1}-x_{i}\right)+2\left(x_{i+1}-x\right)\right]}{\left(x_{i+1}-x_{i}\right)^{3}}\\
&\gamma_{i}(x)=\dfrac{\left(x-x_{i}\right)\left(x_{i+1}-x\right)^{2}}{\left(x_{i+1}-x_{i}\right)^{2}}, \; \delta_{i+1}(x)=\dfrac{\left(x-x_{i}\right)^2\left(x-x_{i+1}\right)}{\left(x_{i+1}-x_{i}\right)^{2}}
\end{aligned}
\end{equation}
Այսպիսով $\left[x_{0}, x_{N}\right]$ հատվածում կտոր առ կտոր մոտարկող ֆունկցիան ներկայացվում է բազիսային ֆունկցիաների գծային կոմբինացիայի տեսքով.
\begin{equation}
p_{3}(x)=\sum_{i=0}^{N}\left[\varphi_{i}^{(0)}f(x_{i})+\varphi_{i}^{(1)}f^{'}(x_{i})\right]
\end{equation}
որտեղ 

\begin{equation}
\begin{aligned}
\varphi^{(0)}_{0}\left(x\right)&=\begin{cases}
\dfrac{\left(x_{1}-x\right)^{2}\left[\left(x_{1}-x_{0}\right)+2\left(x-x_{0}\right)\right]}{\left(x_{1}-x_{0}\right)^{3}},  x\in \left[x_{0}, x_{1}\right]\\
0, x\in \left[x_{1}, x_{N}\right]\\
\end{cases}\\
\varphi^{(0)}_{i}\left(x\right)&=\begin{cases}
0, x\in \left[x_{0}, x_{i-1}\right]\\
\dfrac{\left(x-x_{i-1}\right)^{2}\left[\left(x_{i}-x_{i-1}\right)+2\left(x_{i}-x\right)\right]}{\left(x_{i}-x_{i-1}\right)^{3}}, x\in \left[x_{i-1}, x_{i}\right]\\
\dfrac{\left(x_{i+1}-x\right)^{2}\left[\left(x_{i+1}-x_{i}\right)+2\left(x-x_{i}\right)\right]}{\left(x_{i+1}-x_{i}\right)^{3}}, x\in \left[x_{i}, x_{i+1}\right]\\
0, x\in \left[x_{i+1}, x_{N}\right]\\
\end{cases}\\
\varphi^{(0)}_{N}\left(x\right)&=\begin{cases}
0, x\in \left[x_{0}, x_{N-1}\right]\\
\dfrac{\left(x-x_{N-1}\right)^{2}\left[\left(x_{N}-x_{N-1}\right)+2\left(x_{N}-x\right)\right]}{\left(x_{N}-x_{N-1}\right)^{3}}, x\in \left[x_{N-1}, x_{N}\right]\\
\end{cases}
\end{aligned}
\end{equation}

\begin{equation}
\begin{aligned}
\varphi^{(1)}_{0}\left(x\right)&=\begin{cases}
\dfrac{\left(x-x_{0}\right)\left(x_{1}-x\right)^{2}}{\left(x_{1}-x_{0}\right)^{2}}, x\in \left[x_{0}, x_{1}\right]\\
0, x\in \left[x_{1}, x_{N}\right]\\
\end{cases}\\
\varphi^{(1)}_{i}\left(x\right)&=\begin{cases}
0, x\in \left[x_{0}, x_{i-1}\right]\\
\dfrac{\left(x-x_{i-1}\right)^2\left(x-x_{i}\right)}{\left(x_{i}-x_{i-1}\right)^{2}}, x\in \left[x_{i-1}, x_{i}\right]\\
\dfrac{\left(x-x_{i}\right)\left(x_{i+1}-x\right)^{2}}{\left(x_{i+1}-x_{i}\right)^{2}}, x\in \left[x_{i}, x_{i+1}\right]\\
0, x\in \left[x_{i+1}, x_{N}\right]\\
\end{cases}\\
\varphi^{(1)}_{N}\left(x\right)&=\begin{cases}
0, x\in \left[x_{0}, x_{N-1}\right]\\
\dfrac{\left(x-x_{N-1}\right)^{2}\left(x-x_{N}\right)}{\left(x_{N}-x_{N-1}\right)^{2}}, x\in \left[x_{N-1}, x_{N}\right]\\
\end{cases}
\end{aligned}
\end{equation}

\begin{figure}[h]
\centering
\includegraphics[width=0.9\textwidth]{images/one_var_quadratic}
\captionsetup{labelformat=empty}
\caption{Նկար 1.2. երկրորդ կարգի էրմիթյան մոտարկող բազիսային ֆունկցիաներ։}
\end{figure}
Վերը ներկայացված մոտարկումների օրինակները էրմիթյան մոտարկման մասնավոր դեպքեր են համապատասխանաբար 1 և 2 կարգերի դեպքում։
Ընդհանուր դեպքում m-րդ կարգի էրմիթյան մոտարկման պայմանը կարելի է գրել հետևյալ կերպ.
\begin{equation}
\dfrac{d^{k}}{dx^{k}}f\left(x_{i}\right)=\dfrac{d^{k}}{dx^{k}}p_{2m-1}\left(x_{i}\right), \;  i=\overline{0, N}, \;  k=\overline{0, m-1}
\end{equation}
\newpage

\subsection*{{\addfontfeatures{FakeBold=2.0}1.2 Քառակուսային և խորանարդային մոտարկում}}
Խնդիրներում, որտեղ անհրաժեշտ է որոշել միայն տրված ֆունկիցիան, ֆունկցիայի ածանցյալներն մոտարկելու փոխարեն դրվում է դրանց անընդհատության պայման հանգուցային կետերում,  բավականին հեշտացնելով դրված խնդիրը և դրա լուծումը։
Նման տիպի մոտարկման կառուցման պարզագույն օրինակը հետևյալն է.
Յուրաքանչյուր $\left[x_{i}, x_{i+1}\right]$ ինտերվալում կառուցվում է այնպիսի պարաբոլ, որ բոլոր $x_{i}$ հանգուցային կետերում առանջին կարգի ածանցյալները լինեն անընդհատ։
\begin{equation}
S_{2}^{(i)}(x)=f(x_{i})+\dfrac{f(x_{i+1})-f(x_{i})}{x_{i+1}-x_{i}}\left(x-x_{i}\right)+c_{i}\left(x-x_{i}\right)\left(x-x_{i+1}\right)
\end{equation}
Ածանցյալների անընդհատության պայմանից կհետևի, որ
\begin{equation}
c_{i}+c_{i-1}=\dfrac{1}{h^{2}}\left(f(x_{i+1})-2f(x_{i})+f(x_{i-1})\right) \; i=\overline{1, N-1}
\end{equation}
Քանի որ համակարը պարունակում է $N-1$ հավասարում, ապա մնում է մեկ ազատ գործակից, որը կարելի գտնել, որևէ $x_{j}$ հանգուցային կետում որոշելով $S_{2}^{(j)''}$-ն։

Առավել կիրառելի են խորանարդային սփլայնները։ Այս դեպքում յուրաքանչյուր $\left[x_{i}, x_{i+1}\right]$ ինտերվալում կառուցվում են երրորդ աստիճանի բազմանդամներ այնպես, որ հանգույցներում առաջին և երկրորդ կարգի ածանցյալները լինեն անընդհատ։ 
Բազմանդամը դիտարկելու փոխարեն դիտարկենք նրա երկրորդ կարգի ածանցյալը: Այն գծային ֆունկցիա է, հետևաբար այն կարելի է ներկայացնել հետևյալ տեսքով.
\begin{equation}
S^{(i)''}_3(x)=c_{i}\dfrac{x_{i+1}-x}{x-x_{i}}+c_{i+1}\dfrac{x-x_{i}}{x_{i+1}-x_{i}}
\end{equation}
\noindent որտեղ $c_{i}$ և $c_{i+1}$-ը $x_{i}$ և $x_{i+1}$ կետերում երկրորդ կարգի ածանցյալների արժենքներն են։ 
\noindent Հաշվի առնելով մոտարկման և անընդհատության պայմանները.

\begin{equation}
\begin{cases}
				S_{3}^{(i)}(x_{i}) &=f_{i}\\
				S_{3}^{(i)}(x_{i+1}) &= f_{i+1}\\
				S_{3}^{(i-1)'}(x_{i}) &= S_{3}^{(i)'}(x_{i})

\end{cases}
\end{equation}
կստանանք հետևյալ տեսքի խորանարդային սփլայն.
\begin{equation}
	S_{3}^{(i)}(x) = \dfrac{c_{i}}{6h}\left(x_{i+1}-x\right)^{3}+\dfrac{c_{i+1}}{6h}\left(x-x_{i}\right)^{3}+\left(\dfrac{f_{i}}{h}-\dfrac{hc_{i}}{6}\right)\left(x_{i+1}-x\right)+\left(\dfrac{f_{i+1}}{h}-\dfrac{hc_{i+1}}{6}\right)\left(x-x_{i}\right)
\end{equation}
որտեղից $c_{i}$ գործակիցների համար կստացվի հետևյալ հավասարումների համակարգը.
\begin{equation}
\begin{cases}
c_{i+1}+4c_{i}+c_{i-1} = \dfrac{6}{h^2}\left(f_{i+1}-2f_{i}+f_{i-1}\right) \; i=\overline{1, N-1}\\
c_{0}=A\\
c_{N}=B
\end{cases}
\end{equation}
\begin{figure}[h!]
\centering
\includegraphics[width=1.0\textwidth]{images/quadratic_and_cubic_interploation}
\captionsetup{labelformat=empty}
\caption{Նկար 1.3. Քառակուսային և խորանարդային մոտարկումների համեմատում։}
\end{figure}
\newpage
Այժմ դիտարկենք լոկալ կրիչ ունեցող խորանարդային սփլայնները։ Այդպիսի սփլայններ առաջարկել է ավստրիացի մաթեմատիկոս Առնոլդ Շնեբերգը։
Դիտարկենք հետևյալ ֆունկցիան.
\begin{equation}
				\gamma(x)=\dfrac{1}{4}\left[\left\{x+2\right\}^{3}_{+} - 4\left\{x+1\right\}^{3}_{+}+ 6\left\{x\right\}^{3}_{+} -4\left\{x-1\right\}^{3}_{+} + \left\{x-2\right\}^{3}_{+}\right]
\end{equation}
որտեղ
\begin{equation}
\left\{x\right\}_{+}=
\begin{dcases}
x, x > 0 \\
0, x \leq 0	
\end{dcases}
\end{equation}
\begin{figure}[H]
\centering
\includegraphics[width=1.0\textwidth]{images/cubic_compact_support_basis}
\captionsetup{labelformat=empty}
\caption{Նկար 1.4. $\gamma$(x) ֆունկցիան և նրա մինչև երրորդ կարգի ածանցյալները։}
\end{figure}
Ենթադրենք տրված է $\left[x_{0}, x_{N}\right]$ ինտերվալը տրոհված $N$ հավասար մասերի h քայլով։ Յուրաքանչյուր $x_{i}$ հանգույցի $(i=\overline{2, N-2})$ համար բազիսային ֆունկցիան տրվում է հետևյալ բանաձևով.
\begin{equation}
B_{i}\left(\dfrac{x}{h}\right)=\gamma \left(\frac{x-x_{0}}{h}-i\right)
\end{equation}
Այս ֆունկցիաները և նրանց մինչև երկրորդ կարգի ածանցյալները հավասար են զրոյի $\mathbb{R} \textbackslash \left(x_{i-2}, x_{i+2}\right)$ տիրույթում։
Մնացած բազիսային ֆունկիաները պետք է կառուցել այլ կերպ, քանի որ դրանց մի մասը դուրս է ընկած  $\left[x_{0}, x_{N}\right]$ ինտերվալից։
\begin{equation}
\begin{aligned}
&B_{0}\left(\frac{x}{h}\right) = \gamma \left(\frac{x}{h}\right) + \left\{\dfrac{h - x}{4h}\right\}^{3}_{+} , \; x \in \left[x_{0}, x_{2}\right] \\
&B_{1}\left(\frac{x}{h}\right) = \gamma \left(\frac{x}{h}\right), \; x \in \left[x_{0}, x_{3}\right]\\
&B_{N-1}\left(\frac{x}{h}\right) = \gamma \left(\frac{x}{h}\right), \; x \in \left[x_{N-3}, x_{N}\right] \\
&B_{N}\left(\frac{x}{h}\right) = \gamma \left(\frac{x}{h}\right) + \left\{\dfrac{x - (N-1)h}{4h}\right\}^{3}_{+} , \; x \in \left[x_{N-2}, x_{N}\right]
\end{aligned}
\end{equation}
\begin{figure}[H]
\centering
\includegraphics[width=1.0\textwidth]{images/all_cubic_compact_support_basis}
\captionsetup{labelformat=empty}
\caption{Նկար 1.5. $B_{j}$ բազիսային ֆունկցիաների գրաֆիկական ներկայացում։}
\end{figure}
Տրված $\left\{\left(x_{j}, f_{j}\right)\right\}_{j=0}^{N}$ մոտարկման տվյալնեով կառուցենք սփլայն.
\begin{equation}
F(x) = \sum_{i=0}^{N} \alpha_{i}B_{i}\left(\frac{x}{h}\right)$$
\end{equation}
Հաշվի առնելով մոտարկման պայմանները կստանանք հետևյալ հավասարումների համակարգը.
\begin{equation}
\begin{dcases}
\frac{5}{4}\alpha_{0} + \frac{1}{4}\alpha_{0} = f_{0}\\
\frac{1}{4}\alpha_{j-1} + \alpha_{j} + \frac{1}{4}\alpha_{j+1} = f_{j}, \; j=\overline{1, N-1}\\
\frac{5}{4}\alpha_{N-1} + \frac{1}{4}\alpha_{N} = f_{N}\\
\end{dcases}
\end{equation}
\newpage
\section*{\centering {\addfontfeatures{FakeBold=2.0}Գլուխ 2 \\ Երկչափ մոտարկում}}
%This command resets the equation's counter 	
\setcounter{equation}{0}
Այժմ դիտարկենք երկու փոփոխականի ֆունկցիայի մոտարկման խնդիրը: Ի տարբերություն մեկ փոփոխականի ֆունկցիայի մոտարկման, այս դեպքում տիրույթի տրոհումը կարելի է իրականացնել կամայական ձևով։

\vspace{1.5mm}
\noindent {\addfontfeatures{FakeBold=2.0}{Սահմանում.}}

\noindent Կասենք, որ տրված $f\left(x, y\right)$ ֆունկցիան  $C^{\left(p, q\right)}$ կարգի ֆունկցիա է, եթե 
\begin{equation*}
\dfrac{\partial^{i+j} f}{\partial x^{i} \partial y^{j}}, \; i = \overline{0, p} , \; j = \overline{0, q}
\end{equation*}
ֆունկցիաները անընդհատ են։

Կախված տրված տիրույթից և դրա տրոհման ձևից, առանձնացվում են հետևյալ մոտակման ձևերը.
\subsection*{{\addfontfeatures{FakeBold=2.0}2.1. Ուղղանկյուն տիրույթ}}
Դիցուք տրված են $f:\Omega\mapsto \Theta$,  $\Theta \subset \mathbb{R}$, $\Omega \subset \mathbb{R}^{2} = \left[x_{0}, x_{M}\right] \times \left[y_{0}, y_{N}\right]$  ուղղանկյուն տիրույթը, որը տրոհված է $\left[x_{i}, x_{i+1}\right] \times \left[y_{j}, y_{j+1}\right]$, ուղղանկյուն էլեմենտների:
$$x_{i+1}-x_{i}=h_{1}, \; y_{j+1}-y_{j}=h_{2}, \; i=\overline{0, M-1}, j=\overline{0, N-1}$$
\begin{figure}[h!]
\centering
\includegraphics[width=0.6\textwidth]{images/two_var_linear}
\captionsetup{labelformat=empty}
\caption{Նկար 2.1. Ուղղանկյունաձև տիրույթի տրոհում ուղղանկյուն էլեմենտների։}
\end{figure}
\newpage
\subsubsection*{{\addfontfeatures{FakeBold=2.0}2.1.1. Էրմիթյան մոտարկում}}
{\addfontfeatures{FakeBold=2.0}Երկգծային մոտարկում}։

Յուրաքանչյուր  $\left[x_{i}, x_{i+1}\right] \times \left[y_{j}, y_{j+1}\right]$ էլեմենտի վրա $f$ ֆունկցիան մոտարկվում է հետևյալ քառակուսային ֆունկցիայով.
\begin{equation}
p_{1}^{(i, j)}(x, y)=f_{ij}\alpha_{i,j}(x,y)+f_{i+1,j}\beta_{i+1,j}(x,y)+f_{i,j+1}\gamma_{i,j+1}(x,y)+f_{i+1, j+1}\delta_{i+1, j+1}(x,y)
\end{equation}
որտեղ 
\begin{equation}
\begin{aligned}
&\alpha_{i,j}(x,y)=\dfrac{1}{h_{1}h_{2}}\left(x_{i+1}-x\right)\left(y_{j+1}-y\right) \\
&\beta_{i+1,j}(x,y)=\dfrac{1}{h_{1}h_{2}}\left(x-x_{i}\right)\left(y_{j+1}-y\right) \\
&\gamma_{i,j+1}(x,y)=\dfrac{1}{h_{1}h_{2}}\left(x_{i+1}-x\right)\left(y-y_{j}\right) \\
&\delta_{i+,j+1}(x,y)=\dfrac{1}{h_{1}h_{2}}\left(x-x_{i}\right)\left(y-y_{j}\right)
\end{aligned}
\end{equation}
Այսպիսով $\left[x_{0}, x_{M}\right] \times \left[y_{0}, y_{M}\right]$ տիրույթում կտոր առ կտոր մոտարկող ֆունկցիան տրվում է հետևյալ բանաձևով.
\begin{equation}
p_{1}(x,y)=\sum_{i=0}^{M}\sum_{j=0}^{N} f_{ij} \varphi^{(ij)}(x,y)
\end{equation}
որտեղ
\begin{equation}
\varphi^{(ij)}\left(x, y\right)=\begin{dcases}
\dfrac{1}{h_{1}h_{2}}\left(x-x_{i-1}\right)\left(y-y_{j-1}\right), (x,y)\in \left[x_{i-1}, x_{i}\right]\times\left[y_{j-1}, y_{j}\right]\\
\dfrac{1}{h_{1}h_{2}}\left(x-x_{i-1}\right)\left(y_{j+1}-y\right), (x,y)\in \left[x_{i-1}, x_{i}\right]\times\left[y_{j}, y_{j+1}\right]\\
\dfrac{1}{h_{1}h_{2}}\left(x_{i+1}-x\right)\left(y-y_{j-1}\right), (x,y)\in \left[x_{i}, x_{i+1}\right]\times\left[y_{j-1}, y_{j}\right]\\
\dfrac{1}{h_{1}h_{2}}\left(x_{i+1}-x\right)\left(y_{j+1}-y\right), (x,y)\in \left[x_{i}, x_{i+1}\right]\times\left[y_{j}, y_{j+1}\right]\\
0, \text{մնացած դեպքերում}
\end{dcases}
\end{equation}
Պարզ է, որ ստացված բազիսային ֆունկցիան $C^{(0, 0)}$ ֆունկցիա է։

\begin{figure}[H]
\centering
\includegraphics[width=0.6\textwidth]{images/bilinear_basis_function}
\captionsetup{labelformat=empty}
\caption{Նկար 2.2. Երկգծային մոտարկման բազիսային ֆունկցիայի գրաֆիկ։}
\end{figure}
\noindent {\addfontfeatures{FakeBold=2.0}Կուրանտի գծային ֆունկցիա}

Այժմ դիտարկենք ուղղանկյուն տիրույթի տրոհման և բազիսային ֆունկցիաների կառուցման այլ տարբերակ։ Այս դեպքում տիրույթը տրոհենք ինչպես նախորդ դեպքում, ի հավելումն յուրաքանչյուր ուղղանկյուն էլեմենտ տրոհելով երկու ուղղանկյուն եռանկյունների, ինչպես ցույց է տրված նկ. 2.3-ում։
Դիտարկենք հետևյալ ֆունկցիան.
\begin{equation}
$$\varphi \left(x,y\right)=\begin{cases}
1-y, &(x,y) \in S_{1} \\
1+x-y, &(x,y) \in S_{2} \\
1+x, &(x,y) \in S_{3} \\
1+y, &(x,y) \in S_{4} \\
1-x+y, &(x,y) \in S_{5} \\
1-x, &(x,y) \in S_{6}\\
0, \text{մնացած դեպքերում}\\
\end{cases}
\end{equation}
\begin{figure}[H]
  \centering
  \begin{minipage}[b]{0.4\textwidth}
    \includegraphics[width=\textwidth]{images/two_var_courant_1}
    \captionsetup{labelformat=empty}
    \caption{Նկար 2.3. Ուղղանկյուն էլեմենտի տրոհումը եռանկյունների։}
  \end{minipage}
  \hfill
  \begin{minipage}[b]{0.4\textwidth}
    \includegraphics[width=\textwidth]{images/two_var_courant_2}
    \captionsetup{labelformat=empty}
    \caption{Նկար 2.4. Բազիսային ֆունկցիայի տեսքը։}
  \end{minipage}
\end{figure}
Պարզ է, որ այն $\left[-1, 1\right] \times \left[-1	, 1\right]$ տիրույթում $C^{(0, 0)}$ ֆունկցիա է: Այն հայտնի է որպես Կուրանտի բազիսային ֆունկցիա, ի պատիվ գերմանացի մաթեմատիկոս Ռիխարդ Կուրանտի։
Այժմ յուրաքանչյուր $\left(x_{i}, y_{j}\right)$ հանգուցային կետի համար կառուցենք բազիսային ֆունկցիա օգտվելով $\left(6\right)$-ից.
\begin{equation}
\varphi^{(ij)}(x,y)=\varphi \left(\dfrac{x-x_{i}}{h_{1}}, \dfrac{y-y_{j}}{h_{2}}\right)
\end{equation}
Յուրաքանչյուր կետի համար ունենալով բազիսային ֆունկցիա, $f$ ֆունկցիայի մոտարկման բանաձևը կտվրի հետևյալ կերպ.
\begin{equation}
p_{1}(x,y)=\sum_{i=0}^{N}\sum_{j=0}^{M}\varphi^{(ij)}(x,y)f(x_{i}, y_{j})
\end{equation}
\begin{figure}[H]
\centering
\includegraphics[width=0.6\textwidth]{images/two_var_courant_3}
\captionsetup{labelformat=empty}
\caption{Նկար 2.5. Տիրույթի տրոհումը ուղղանկյուն եռանկյունների։}
\end{figure}

\newpage
{\addfontfeatures{FakeBold=2.0}Երկխորանարդային մոտարկում}

Միավոր քառակուսու վրա կառուցենք մոտարկող ֆունկցիա, որը լրիվ խորանարդային բազմանդամ է ըստ երկու փոփոխականի.
\begin{equation}
			p_{3}^{(i, j)}(x, y)=\sum_{i, j=0}^{3}\alpha_{ij}x^{i}y^{j}
\end{equation}
որտեղ $\alpha_{ij}$ գործակիցները միարժեքորեն որոշվում են գագաթների վրա $f, \dfrac{\partial f}{\partial x}, \dfrac{\partial f}{\partial y}, \dfrac{\partial^{2} f}{\partial x \partial y}$ ֆունկցիաների արժեքներով։ Դրանք որոշելով $\left(8\right)$ գումարը կարող ենք ներկայացնել հետևյալ տեսքով.
\begin{equation}
			p_{3}^{(i, j)}(x, y)= \mathlarger{\sum_{0 \leq i, j, k, l \leq 1}} \dfrac{\partial^{k+l}}{\partial x^{k} \partial y^{l}}  \psi_{i}^{(k)}(x)\psi_{j}^{(l)}(y)
\end{equation}
որտեղ
\begin{equation*}
\begin{aligned}
&\psi_{0}^{(0)}(t) = \left(1-t\right)^{2}\left(1+2t\right) \\
&\psi_{0}^{(1)}(t) = \left(1-t\right)^{2}t \\
&\psi_{1}^{(0)}(t) = t^{2}\left(3-2t\right) \\
&\psi_{1}^{(1)}(t) = t^{2}\left(t-1\right) \\
\end{aligned}
\end{equation*}
Այժմ օգտվելով $\left(8\right)$ բանաձևից և ունենալով տրված տիրույթի տրոհման $h_{1}$ և $h_{2}$ քայլերը, յուրաքանչյուր գագաթի համար կարող ենք կազմել բազիսային ֆունկցիա, և ամբողջ տիրույթում մոտարկող ֆունկցիան կունենա հետևյալ տեսքը.
\begin{equation}
p_{3}(x)=\sum_{i=0}^{M}\sum_{j=0}^{N}\left[f_{ij}\alpha^{(ij)}\left(x,y\right)+\dfrac{\partial f_{ij}}{\partial x}\beta^{(ij)}\left(x,y\right) +\dfrac{\partial f_{ij}}{\partial y}\gamma^{(ij)}\left(x,y\right) +\dfrac{\partial^{2} f_{ij}}{\partial x\partial y}\delta^{(ij)}\left(x,y\right)\right]
\end{equation}
որտեղ
\begin{equation}
\begin{aligned}
&\alpha^{(ij)} = \varphi^{(0)}\left(\dfrac{x-x_{0}}{h_{1}}-i\right) \varphi^{(0)}\left(\dfrac{y-y_{0}}{h_{2}}-j\right),  \beta^{(ij)} = \varphi^{(1)}\left(\dfrac{x-x_{0}}{h_{1}}-i\right) \varphi^{(0)}\left(\dfrac{y-y_{0}}{h_{2}}-j\right), \\
&\gamma^{(ij)} = \varphi^{(0)}\left(\dfrac{x-x_{0}}{h_{1}}-i\right) \varphi^{(1)}\left(\dfrac{y-y_{0}}{h_{2}}-j\right), \delta^{(ij)} = \varphi^{(1)}\left(\dfrac{x-x_{0}}{h_{1}}-i\right) \varphi^{(1)}\left(\dfrac{y-y_{0}}{h_{2}}-j\right)\\
&\varphi^{(0)}=\begin{dcases}
\left(1+t\right)^{2}\left(1-2t\right), \; -1\leq x \leq 0 \\
\left(1-t\right)^{2}\left(1+2t\right), \; 0\leq  x \leq  1 \\
\end{dcases}, \varphi^{(1)}=\begin{dcases}
\left(1+t\right)^{2}t, \; -1\leq  x \leq 0 \\
\left(1-t\right)^{2}t, \; 0\leq  x \leq 1 \\
\end{dcases}
\end{aligned}
\end{equation}
\begin{figure}[H]
  \centering
  \begin{minipage}[b]{0.45\textwidth}
    \includegraphics[width=\textwidth]{images/two_dimensional_ermite_1}
    %\captionsetup{labelformat=empty}
  \end{minipage}
  \hfill
  \begin{minipage}[b]{0.45\textwidth}
    \includegraphics[width=\textwidth]{images/two_dimensional_ermite_2}
    %\captionsetup{labelformat=empty}
  \end{minipage}
\\
  \begin{minipage}[b]{0.45\textwidth}
    \includegraphics[width=\textwidth]{images/two_dimensional_ermite_3}
    %\captionsetup{labelformat=empty}
  \end{minipage}
\hfill
  \begin{minipage}[b]{0.45\textwidth}
    \includegraphics[width=\textwidth]{images/two_dimensional_ermite_4}
    %\captionsetup{labelformat=empty}
  \end{minipage}
\captionsetup{labelformat=empty}
\caption{Նկար 2.6 Երկխորանարդային մոտարկման բազիսային ֆունկցիաների գծապատկեր։}
\end{figure}
Նկատենք, որ մոտարկումը կարելի էր ստանալ նաև միաչափ խորանարդային էրմիթյան բազիսային ֆունկցիաների թենզորական արտադրյալի միջոցով։

Վերը ներկայացված երկու մոտարկումները էրմիթյան մոտարկման մասնովոր դեպքեր են։
Ընդհանուր դեպքում ցանկացած $k$ բնական թվի և տրված ուղղանկյունաձև տրոհման յուրաքանչյուր էլեմենտի վրա կարելի է կառուցել $C^{\left(k-1, k-1\right)}$ մոտարկող ֆունկցիա, որը ըստ յուրաքանչյուր փոփոխականի $\left(2k-1\right)$-րդ կարգի բազմանդամ է և բավարարում է հետևյալ պայմաններին.
\begin{equation}
\dfrac{\partial^{p+q}}{\partial x^{p} \partial y^{q}}f\left(x_{i}, y_{j}\right)=\dfrac{\partial^{p+q}}{\partial x^{p} \partial y^{q}}p_{2k-1}\left(x_{i}, y_{j}\right), \; p,q=\overline{0, k-1}, \;  i=\overline{0, M}, \; j=\overline{0, N}
\end{equation}

\newpage
\subsubsection*{{\addfontfeatures{FakeBold=2.0}2.1.2 Խորանարդային մոտարկում: Երկխորանարդային սփլայն}}

Մեկ փոփոխականի մոտարկան խնդիրը դիտարկելիս քննարկեցինք Շներբերգի կողմից առաջարկված խորանարդային բազիսային սփլայնները։ Այս սփլայնները կարող ենք  օգտագործել երկչափ մոտարկման համար։
Յուրաքանչյուր $\left(x_{i}, y_{j}\right)$ հանգույցի վրա կառուցենք բազիսային ֆունկցիա կազմելով $B_{i}\left(\dfrac{x}{h_{1}}\right)$ և $B_{j}\left(\dfrac{x}{h_{2}}\right)$ ֆունկցիաների թենզորական  արտադրյալը։
\begin{equation}
\varphi^{(ij)}\left(x, y\right) = B_{i}\left(\dfrac{x}{h_{1}}\right) B_{j}\left(\dfrac{x}{h_{2}}\right)$$
\end{equation}
Ստացված բազիսային ֆունկցիան $C^{(2, 2)}$ ֆունկցիա է։

Տրված $\left\{\left(x_{ij}, y_{ij}, f_{ij}\right)\right\}$ մոտարկման տվյալներով կառուցենք սփլայն.
\begin{equation}
F(x,y) = \sum_{i=0}^{N} \alpha_{ij}\varphi^{(ij)}\left(x, y\right)$$
\end{equation}

Հաշվի առնելով մոտարկման պայմանները կստանանք հետևյալ հավասարումերի համակարգը.
\begin{equation}
\begin{dcases}
\alpha_{00}+\frac{5}{16}(\alpha_{01} + \alpha_{10}) +\frac{1}{16}\alpha_{11} = f_{00}\\
\alpha_{M0}+\frac{5}{16}(\alpha_{M1} + \alpha_{M-1,0}) +\frac{1}{16}\alpha_{M-1, 1} = f_{M0}\\
\alpha_{0N}+\frac{5}{16}(\alpha_{1N} + \alpha_{0,N-1}) +\frac{1}{16}\alpha_{N-1, 1} = f_{0N}\\
\alpha_{MN}+\frac{5}{16}(\alpha_{M-1,N} + \alpha_{M,N-1}) +\frac{1}{16}\alpha_{M-1,N-1} = f_{MN}\\
\frac{5}{4}\alpha_{i,1}+\frac{5}{16}(\alpha_{i-1,0} + \alpha_{i+1,0}) +\frac{1}{4}\alpha_{i,1} + \frac{1}{16}(\alpha_{i-1,1}+\alpha_{i+1,1}) = f_{i0}\\
\frac{5}{4}\alpha_{i,N-1}+\frac{5}{16}(\alpha_{i-1,N} + \alpha_{i+1,N}) +\frac{1}{4}\alpha_{i,N-1} + \frac{1}{16}(\alpha_{i-1,N-1}+\alpha_{i+1,N-1}) = f_{iN}\\
\frac{5}{4}\alpha_{1,j}+\frac{5}{16}(\alpha_{0,j-1} + \alpha_{0,j+1}) +\frac{1}{4}\alpha_{1,j} + \frac{1}{16}(\alpha_{1,j-1}+\alpha_{1,j+1}) = f_{0j}\\
\frac{5}{4}\alpha_{M-1,j}+\frac{5}{16}(\alpha_{M,j-1} + \alpha_{M,j+1}) +\frac{1}{4}\alpha_{M-1,j} + \frac{1}{16}(\alpha_{M-1,j-1}+\alpha_{M-1,j+1}) = f_{Mj}\\
\alpha_{ij} + \frac{1}{4}(\alpha_{i,j-1} + \alpha_{i,j+1} + \alpha_{i-1,j} + \alpha_{i+1,j}) + \frac{1}{16}(\alpha_{i-1,j-1} + \alpha_{i-1,j+1} + \alpha_{i+1,j-1} +\alpha_{i+1,j+1}) = f_{ij}
\end{dcases}
\end{equation}
\newpage
\begin{figure}[h!]
\centering
\includegraphics[width=1.0\textwidth]{images/two_dimensional_basis}
\captionsetup{labelformat=empty}
\caption{Նկար 2.7. $\varphi^{(ij)}$ բազիսային ֆունկցիաների պատկեր։}
\end{figure}
\begin{figure}[h!]
\centering
\includegraphics[width=1.0\textwidth]{images/two_dimensional_basis_1}
\captionsetup{labelformat=empty}
\caption{Նկար 2.8 Տիրույթի եզրի վրա գտնվող բազիսային ֆունկցիա։}
\end{figure}
\newpage
\subsection*{{\addfontfeatures{FakeBold=2.0}2.2 Բազմանկյուն տիրույթ}}
Բազմանկյուն տիրույթ ասելով կհասկանանք կամ հենց բազմանկյունաձև տիրույթը, կամ դրա բազմանկյուններով մոտարկումը։
Դիցուք տրված են $f:\Omega\mapsto \Theta, \Theta \subset \mathbb{R}, \Omega \subset \mathbb{R}^{2} $ բազմանկյուն տիրույթը, որը կամայական ձևով տրոհված է եռանկյուն էլեմենտների։
\begin{figure}[H]
\centering
\includegraphics[width=0.4\textwidth]{images/two_var_triangular}
\captionsetup{labelformat=empty}
\caption{Նկար 2.9. Տիրույթի եռանկյունաձև տրոհման օրինակ։}
\end{figure}
\subsubsection*{{\addfontfeatures{FakeBold=2.0}2.2.1 Լագրանժի մոտարկում}}
Յուրաքանչուր եռանկյուն էլեմենտի վրա դիտարկենք $m$-րդ կարգի լրիվ բազմանդամ
\begin{equation}
				F_{m}\left(x,y\right)=\sum_{k+l=0}^{m}\alpha_{kl}x^{k}y^{l}
\end{equation}
որը կարող է օգտագործվել որպես մոտարկող ֆունկցիա $\dfrac{1}{2}\left(m+1\right)\left(m+2\right)$ սիմետրիկ դասավորված կետերի վրա։
Դիտարկենք հետևյալ մասնավոր դեպքերը.

%This command allows to indent only the first line of enumeration
\setlist[enumerate]{itemindent=\dimexpr\labelwidth+\labelsep\relax,leftmargin=0pt}

\begin{enumerate}[leftmargin=0.0cm]
\item{Գծային մոտարկում (m=1)}

Յուրաքանչյուր $P_{1}, P_{2}, P_{3}$ գագաթներով եռանկյուն էլեմենտի համար դիտարկենք հետևյալ մոտարկող ֆունկցիան.
\begin{equation}
F_{1}(x, y) = \dfrac{1}{S}\sum_{i=1}^{3} p^{(1)}_{i}(x,y)f(x_{i}, y_{i})
\end{equation}
որտեղ $S$-ը $P_{1}, P_{2}, P_{3}$ կետերով կազմված եռանկյան մակերեսի կրկնապատիկն է, իսկ $p^{(1)}_{i}$ ֆունկցիաները սահմանվում են հետևյալ կերպ.
\begin{equation}
p^{(1)}_{i}(x,y) = x_{j}y_{k}-x_{k}y_{j} + \eta_{kl}x - \xi_{kl}y
\end{equation}
որտեղ
			$$\eta_{kl}=y_{k}-y_{l},\; \xi_{kl} = x_{k}-x_{l}$$
$$S = \begin{vmatrix}
     1 & x_1 & y_1\\ 
     1 & x_2 & y_2\\
     1 & x_3 & y_3 
\end{vmatrix}$$
որտեղ $(x_{i}, y_{i}), \; i=1, 2, 3$ տրված եռանկյուն էլեմենտի գագաթներն են (հերթականությունը ժամսլաքին հակառակ ուղղությամբ)։
\item{Քառակուսային մոտարկում (m=2)}

Յուրաքանչյուր եռանկյուն էլեմենտի վրա գագաթներից բացի դիտարկենք նաև միջնակետերը։ Մոտարկող բազմանդամը ունի հետևյալ տեսքը.
\begin{equation}
F_{2}(x, y) = \dfrac{1}{S^{2}}\sum_{i=1}^{6} p^{(2)}_{i}(x,y)f(x_{i}, y_{i})
\end{equation}
որտեղ 
\begin{equation}
\begin{aligned}
&p^{(2)}_{i}(x,y) = p^{(1)}_{i}(x,y)\left(2 p^{(1)}_{i}(x,y)-1\right), \; i = 1, 2, 3 \\
&p^{(2)}_{4}(x,y) =  4p^{(1)}_{1}(x,y)p^{(1)}_{2}(x,y) \\
&p^{(2)}_{5}(x,y) =  4p^{(1)}_{2}(x,y)p^{(1)}_{3}(x,y) \\
&p^{(2)}_{6}(x,y) =  4p^{(1)}_{1}(x,y)p^{(1)}_{3}(x,y)
\end{aligned}
\end{equation}
\begin{figure}[H]
\centering
\includegraphics[width=0.4\textwidth]{images/quadratic_on_triangular}
\captionsetup{labelformat=empty}
\caption{Նկար 2.8. Կետերի դասավորվածությունը կամայական եռանկյան վրա։}
\end{figure}
\item{Խորանարդային մոտարկում (m=3)}

Այս տեպքում յուրականչյուր կողի վրա ավելացնենք ևս երկու կետ այն տրոհելով երեք հավասար մասի  և դրանց ավելացնելով եռանկյան կենտրոնը։ Մոտարկող բազմանդամը ունի հետևյալ տեսքը.
\begin{equation}
F_{3}(x, y) = \dfrac{1}{S^{3}}\sum_{i=1}^{10} p^{(3)}_{i}(x,y)f(x_{i}, y_{i})
\end{equation}
որտեղ 
\begin{equation}
\begin{aligned}
&p^{(3)}_{i}(x,y) =  \dfrac{1}{2}p^{(1)}_{i}(x,y)\left(3 p^{(1)}_{i}(x,y)-1\right)\left(3 p^{(1)}_{i}(x,y)-2\right), \; i = 1, 2, 3 \\
&p^{(3)}_{4}(x,y) =  \dfrac{9}{2}p^{(1)}_{1}(x,y) p^{(1)}_{2}(x,y)\left(3 p^{(1)}_{1}(x,y)-1\right)\\
&p^{(3)}_{5}(x,y) =  \dfrac{9}{2}p^{(1)}_{1}(x,y) p^{(1)}_{2}(x,y)\left(3 p^{(1)}_{2}(x,y)-1\right)\\
&p^{(3)}_{6}(x,y) =  \dfrac{9}{2}p^{(1)}_{2}(x,y) p^{(1)}_{3}(x,y)\left(3 p^{(1)}_{3}(x,y)-1\right)\\
&p^{(3)}_{7}(x,y) =  \dfrac{9}{2}p^{(1)}_{2}(x,y) p^{(1)}_{3}(x,y)\left(3 p^{(1)}_{3}(x,y)-1\right) \\
&p^{(3)}_{8}(x,y) =  \dfrac{9}{2}p^{(1)}_{3}(x,y) p^{(1)}_{1}(x,y)\left(3 p^{(1)}_{3}(x,y)-1\right)\\
&p^{(3)}_{9}(x,y) =  \dfrac{9}{2}p^{(1)}_{3}(x,y) p^{(1)}_{1}(x,y)\left(3 p^{(1)}_{1}(x,y)-1\right) \\
&p^{(3)}_{10}(x,y) = 27p^{(1)}_{1}(x,y)p^{(1)}_{2}(x,y)p^{(1)}_{3}(x,y)
\end{aligned}
\end{equation}

\begin{figure}[H]
\centering
\includegraphics[width=0.4\textwidth]{images/cubic_on_triangular}
\captionsetup{labelformat=empty}
\caption{Նկար 2.9. Կետերի դասավորվածությունը կամայական եռանկյան վրա։}
\end{figure}

\end{enumerate}
Բնական է, որ որևէ կետի նկատմամբ լրիվ բազիսային ֆունկցիայի կառուցելու համար անհրաժեշտ է իրար գումարել այն բոլոր եռանկյուն էլեմենտների այդ կետին համապատասխան ֆունկցիաները, որոնց համար տվյալ կետը գագաթ է։

\newpage
\noindent{\addfontfeatures{FakeBold=2.0}Ստանդարտ եռանկյուն}

 $\left(2.17\right)$ ից կարելի է հեշտորեն ստանալ, որ 

%This command allows to indent only the first line of enumeration
\setlist[enumerate]{itemindent=\dimexpr\labelwidth+\labelsep\relax,leftmargin=0pt}
\begin{enumerate}[leftmargin=0.0cm]
\item
$\sum_{i=1}^{3}p_{i}\left(x,y\right)=1$
\item
$p_{i}\left(x,y\right)=0$ գծային հավասարումները եռանկյան համապատասխանաբար $P_{2}P_{3}$, $P_{3}P_{1}$ և $P_{1}P_{2}$ կողերն են։
\item
$p_{i}\left(x,y\right)=1$ գծային հավասարումները եռանկյան համապատասխանաբար $P_{1}$, $P_{2}$ և $P_{3}$ գագաթներն են։
\end{enumerate}

Այսինքն, կարող ենք ասել, որ $P_{1}P_{2}P_{3}$ եռանկյունը $(x, y)$ կոորդինատական հարթությունից ձևափոխվում է այսպես կոչված ստանդարտ եռանկյան, որի գագաթներն են $(0, 1)$, $(1, 0)$ և $(0, 0)$ կետերը։ Հակադարձ ձևափոխությունը տրվում է հետևյալ բանաձևերով.
\begin{equation}
\begin{aligned}
&x = x_{3} + \xi_{13}p_{1}+\xi_{23}p_{2}\\
&y = y_{3} + \eta_{13}p_{1}+\eta_{23}p_{2}
\end{aligned}
\end{equation}
Հաշվարկները և բանաձևրը հեշտանցնելու համար հետագայում կօգտվենք այս ձևափոխությունից։
\begin{figure}[H]
\centering
\includegraphics[width=0.8\textwidth]{images/standard_triangle_transformation}
\captionsetup{labelformat=empty}
\caption{Նկար 2.10. Կամայական եռանկյան ձևափոխում ստանդարտ եռանկյան։}
\end{figure}
Նշենք սակայն, որ որոշ գրականություններում հանդիպում են ստանդարտ եռանկյան գագաթների համարակալման այլ հաջորդականություն, բայց անցումը մի եղանակից մյուսը տրիվիալ է։
\newpage
\subsubsection*{{\addfontfeatures{FakeBold=2.0}2.2.2 Էրմիթյան մոտարկում}}
Նախորդ ենթագլխում տեսանք, որ ավելի բարձր կարգի մոտարկում իրականացնելու համար անհրաժեշտ է մեկ եռանկյան վրա դիտարկել մեծ քանակությամբ կետեր։ Որպես այլընտրանք դիտարկենք Էրմիթի մոտարկումը, որի դեպքում դիտարկվող կետերի քանակը ավելի քիչ է։ k-րդ աստիճանի բազմանդամների դասը նշանակենք $G_{k}$-ով։ $\left(2.12\right)$-ի նմանությամբ, այս դեպքում ևս տրված եռանկյուն էլեմենտի վրա ընդհանուր դեպքում $2k+1$-րդ կարգի Էրմիթյան մոտարկման գործակիցները որոշվում են հետևյալ բանաձևերով։
\begin{equation}
\dfrac{\partial^{k} G_{n}}{\partial p_{1}^{k_{1}} \partial p_{2}^{k_{2}}}\left(P_{i}\right), i = \overline{1, 3}\\
\end{equation}
որտեղ $P_{1}, P_{2}, P_{3}$ տրված եռանկյան գագաթներն են։

Դիտարկենք երրորդ աստիճանի (k=1) դեպքը։

Այս դեպքում տրված եռանկյուն ելեմենտի վրա մոտարկող բազմանդամը ունի 9 պարամետր, որոնք միարժեքորեն որոշվում են գագաթներում ֆունկցիայի և նրա առաջին կարգի ածանցյալների արժեքներով։ Ըստ այդմ մոտարկող ֆունկցիան կներկայացնենք հետևյալ տեսքով.
\begin{equation}
				G_{1}\left(p_{1},p_{2}\right)=\sum_{j=1}^{3}\left[U_{j}q^{(3)}_{j}+\left(\dfrac{\partial U}{\partial p_{1}}\right)_{j}r^{(3)}_{j}+\left(\dfrac{\partial U}{\partial p_{1}}\right)_{j}s^{(3)}_{j}\right]
\end{equation}
որտեղ
\begin{equation}
\begin{aligned}
&q_{j}^{(3)}\left(x,y\right)=p_{j}^{2}\left(3 - 2p_{j}\right) + 2p_{1}p_{2}p_{3} \\
&r^{(3)}_{1} = p_{1}^{2}\left(p_{1} - 1\right) - p_{1}p_{2}p_{3} \\
&r^{(3)}_{2} = p_{2}^{2}p_{1} + \dfrac{p_{1}p_{2}p_{3}}{2} \\
&r^{(3)}_{3} = p_{3}^{2}p_{1} + \dfrac{p_{1}p_{2}p_{3}}{2}\\
&s^{(3)}_{1} = p_{1}^{2}p_{2}+\dfrac{p_{1}p_{2}p_{3}}{2}\\
&s^{(3)}_{2} = p_{2}^{2}\left(p_{2}-1\right) -p_{1}p_{2}p_{3}\\
&s^{(3)}_{3} = p_{3}^{2}p_{2}+\dfrac{p_{1}p_{2}p_{3}}{2}
\end{aligned}
\end{equation}
Ստորև ներկայացված են կառուցված բազիսային ֆունկցիաներ հինգ եռանկյուն էլեմենտների վրա $U, \dfrac{\partial U}{\partial p_{1}}, \dfrac{\partial U}{\partial p_{2}}$ հերթականությամբ։

\begin{figure}[H]
  \centering
  \begin{minipage}[b]{0.43\textwidth}
    \includegraphics[width=\textwidth]{images/hermite_cubic_func}
  \end{minipage}
  \hfill
  \begin{minipage}[b]{0.43\textwidth}
    \includegraphics[width=\textwidth]{images/hermite_cubic_fx}
  \end{minipage}
\\
  \begin{minipage}[b]{0.43\textwidth}
    \includegraphics[width=\textwidth]{images/hermite_cubic_fy}
  \end{minipage}
\captionsetup{labelformat=empty}
\caption{Նկար 2.10. Երրորդ կարգի Էրմիթի մոտարկման բազիսային ֆունկցիաները կառուցված հինգ եռանկյունների վրա։}
\end{figure}

Նկատենք սակայն, որ նախորդիվ դիտարկված մոտարկումը $C^{0}$ կարգի է։ Այժմ դիտարկենք $C^{1}$ կարգի մոտարկման մի եղանակ, որն անվանում են Արգիրիսի էլեմենտ։

Այս դեպքում, ստանդարտ եռանյան գագաթների համարակալման հաջորդականությունը հետևյալն է՝ $\left(0,0\right), \left(0,1\right), \left(1,0\right)$։

Ստանդարտ եռանկյան վրա դիտարկենք հիգերորդ աստիճանի լրիվ բազմանդամ, որն ունի 21 պարամետր։
\begin{equation}
				U\left(p_{1}, p_{2}\right) = \sum_{j+k\leq 5} \alpha_{jk}p_{1}^{j}p_{2}^{k}
\end{equation}
$\alpha_{jk}$ պարամետրերը որոշվում են գագաթներում ֆունկցիայի և նրա մինչև երկրորդ կարգի բոլոր ածանցյալների արժեքներով և կողերի միջնակետերում նորմալի ուղղությամբ ածանցյալի արժեքներով։ Միջնակետերը կհամարակելենք հետևյալ կերպ։ $j$-րդ գագաթին համապատասխանող միջնակետին կհամապատասխանեցնենք այն կողի միջնակետը, որին այն չի պատկանում։
\begin{figure}[H]
\centering
\includegraphics[width=0.3\textwidth]{images/argyris_element}
\captionsetup{labelformat=empty}
\caption{Նկար 2.11. Արգիրիսի եռանկյան սխեմատիկ ներկայացում։}
\end{figure}

Ստորև ներկայացվում են բազիսային ֆունկցիաները։

Առաջին գագաթի համար՝
\begin{equation}
\begin{aligned}
&\varphi_{1}^{(0,0)}=- 6 p_{1}^{5} + 15 p_{1}^{4} + 30 p_{1}^{3} p_{2}^{2} - 10 p_{1}^{3} + 30 p_{1}^{2} p_{2}^{3} - 30 p_{1}^{2} p_{2}^{2} - 6 p_{2}^{5} + 15 p_{2}^{4} - 10 p_{2}^{3} + 1 \\
&\varphi_{1}^{(1,0)}=p_{1} \left(- 3 p_{1}^{4} + 8 p_{1}^{3} + p_{1}^{2} p_{2}^{2} - 6 p_{1}^{2} - 10 p_{1} p_{2}^{3} + 10 p_{1} p_{2}^{2} - 8 p_{2}^{4} + 18 p_{2}^{3} - 11 p_{2}^{2} + 1\right) \\
&\varphi_{1}^{(0,1)}=p_{2} \left(- 8 p_{1}^{4} - 10 p_{1}^{3} p_{2} + 18 p_{1}^{3} + p_{1}^{2} p_{2}^{2} + 10 p_{1}^{2} p_{2} - 11 p_{1}^{2} - 3 p_{2}^{4} + 8 p_{2}^{3} - 6 p_{2}^{2} + 1\right) \\
&\varphi_{1}^{(2,0)}=\frac{p_{1}^{2} \left(- p_{1}^{3} + 3 p_{1}^{2} + 3 p_{1} p_{2}^{2} - 3 p_{1} + 2 p_{2}^{3} - 3 p_{2}^{2} + 1\right)}{2} \\
&\varphi_{1}^{(1,1)}=p_{1} p_{2} \left(- 2 p_{1}^{3} - 6 p_{1}^{2} p_{2} + 5 p_{1}^{2} - 6 p_{1} p_{2}^{2} + 10 p_{1} p_{2} - 4 p_{1} - 2 p_{2}^{3} + 5 p_{2}^{2} - 4 p_{2} + 1\right) \\
&\varphi_{1}^{(0,2)}=\frac{p_{2}^{2} \cdot \left(2 p_{1}^{3} + 3 p_{1}^{2} p_{2} - 3 p_{1}^{2} - p_{2}^{3} + 3 p_{2}^{2} - 3 p_{2} + 1\right)}{2} \\
\end{aligned}
\end{equation}
Երկրորդ գագաթի համար՝
\begin{equation}
\begin{aligned}
&\varphi_{2}^{(0,0)}=p_{1}^{2} \cdot \left(6 p_{1}^{3} - 15 p_{1}^{2} - 15 p_{1} p_{2}^{2} + 10 p_{1} - 15 p_{2}^{3} + 15 p_{2}^{2}\right) \\
&\varphi_{2}^{(1,0)}=\frac{p_{1}^{2} \left(- 6 p_{1}^{3} + 14 p_{1}^{2} + 7 p_{1} p_{2}^{2} - 8 p_{1} + 7 p_{2}^{3} - 7 p_{2}^{2}\right)}{2} \\
&\varphi_{2}^{(0,1)}=\frac{p_{1}^{2} p_{2} \left(- 16 p_{1}^{2} - 37 p_{1} p_{2} + 28 p_{1} - 27 p_{2}^{2} + 37 p_{2} - 10\right)}{2} \\
&\varphi_{2}^{(2,0)}=\frac{p_{1}^{2} \cdot \left(2 p_{1}^{3} - 4 p_{1}^{2} - p_{1} p_{2}^{2} + 2 p_{1} - p_{2}^{3} + p_{2}^{2}\right)}{4} \\
&\varphi_{2}^{(1,1)}=\frac{p_{1}^{2} p_{2} \left(4 p_{1}^{2} + 7 p_{1} p_{2} - 6 p_{1} + 5 p_{2}^{2} - 7 p_{2} + 2\right)}{2} \\
&\varphi_{2}^{(0,2)}=\frac{p_{1}^{2} p_{2}^{2} \left(- 3 p_{1} - 5 p_{2} + 5\right)}{4} \\
\end{aligned}
\end{equation}
Երրորդ գագաթի համար՝
\begin{equation}
\begin{aligned}
&\varphi_{3}^{(0,0)}=p_{2}^{2} \left(- 15 p_{1}^{3} - 15 p_{1}^{2} p_{2} + 15 p_{1}^{2} + 6 p_{2}^{3} - 15 p_{2}^{2} + 10 p_{2}\right) \\
&\varphi_{3}^{(1,0)}=\frac{p_{1} p_{2}^{2} \left(- 27 p_{1}^{2} - 37 p_{1} p_{2} + 37 p_{1} - 16 p_{2}^{2} + 28 p_{2} - 10\right)}{2} \\
&\varphi_{3}^{(0,1)}=\frac{p_{2}^{2} \cdot \left(7 p_{1}^{3} + 7 p_{1}^{2} p_{2} - 7 p_{1}^{2} - 6 p_{2}^{3} + 14 p_{2}^{2} - 8 p_{2}\right)}{2} \\
&\varphi_{3}^{(2,0)}=\frac{p_{1}^{2} p_{2}^{2} \left(- 5 p_{1} - 3 p_{2} + 5\right)}{4} \\
&\varphi_{3}^{(1,1)}=\frac{p_{1} p_{2}^{2} \cdot \left(5 p_{1}^{2} + 7 p_{1} p_{2} - 7 p_{1} + 4 p_{2}^{2} - 6 p_{2} + 2\right)}{2} \\
&\varphi_{3}^{(0,2)}=\frac{p_{2}^{2} \left(- p_{1}^{3} - p_{1}^{2} p_{2} + p_{1}^{2} + 2 p_{2}^{3} - 4 p_{2}^{2} + 2 p_{2}\right)}{4} \\
\end{aligned}
\end{equation}
Միջնակետերի համար՝
\begin{equation}
\begin{aligned}
&\hat{\varphi_{1}}=8 \sqrt{2} p_{1}^{2} p_{2}^{2} \left(p_{1} + p_{2} - 1\right) \\
&\hat{\varphi_{2}}=16 p_{1} p_{2}^{2} \left(- p_{1}^{2} - 2 p_{1} p_{2} + 2 p_{1} - p_{2}^{2} + 2 p_{2} - 1\right) \\
&\hat{\varphi_{3}}=16 p_{1}^{2} p_{2} \left(- p_{1}^{2} - 2 p_{1} p_{2} + 2 p_{1} - p_{2}^{2} + 2 p_{2} - 1\right)
\end{aligned}
\end{equation}

Այսպիսով, մոտարկող ֆունկցիան ստանդարտ եռանկյան վրա կունենա հետևյալ տեսքը.
\begin{equation}
U\left(p_{1}, p_{2}\right) = \sum_{i=1}^{3}\sum_{|j|\leq 2}D^{j}U_{i} \varphi_{i}^{j} \left(p_{1}, p_{2}\right) + \sum_{i=1}^{3} \dfrac{\partial U_{i}}{\partial n}\hat{\varphi_{i}}\left(p_{1}, p_{2}\right)
\end{equation}

որտեղ 

$$D^{j}U_{i} = \dfrac{\partial^{|i|}U}{\partial p_{1}^{i_{1}} \partial p_{2}^{i_{2}}}, \; \left|i\right| = i_{1}+i_{2}$$
ածանցյալի մուլտիինդեքսային ներկայացման եղանակն է։

%TODO: Add reference to A general .....
Ստանդարտ եռանկյունից ցանկացած եռանկյան վրա արտապատկերելու համար կօգտվենք Պիոլայի ձևափոխությունից։ Կատարենք ձևափոխության համար անհրաժեշտ սահամները:

$\hat{\Psi}$-ով նշանակենք ստանդարտ եռանկյան բազիսները, իսկ $\Psi$ -ով՝ դրանք արտապարկերված եռանկյան վրա։

Նշանակենք

%This command allow to increase the max number of matrix cols
\setcounter{MaxMatrixCols}{20}
\begin{equation}
V = \begin{bmatrix}
1 & 0 & 0 & 0 & 0 & 0 & 0 & 0 & 0 & 0 & 0 & 0\\
0 & J^{-T} & 0 & 0 & 0 & 0 & 0 & 0 & 0 & 0 & 0 & 0\\
0 & 0 & \Theta^{-1} & 0 & 0 & 0 & 0 & 0 & 0 & 0 & 0 & 0\\
0 & 0 & 0 & 1 & 0 & 0 & 0 & 0 & 0 & 0 & 0 & 0\\
0 & 0 & 0 & 0 & J^{-T} & 0 & 0 & 0 & 0 & 0 & 0 & 0\\
0 & 0 & 0 & 0 & 0 & \Theta^{-1} & 0 & 0 & 0 & 0 & 0 & 0\\
0 & 0 & 0 & 0 & 0 & 0 & 1 & 0 & 0 & 0 & 0 & 0\\
0 & 0 & 0 & 0 & 0 & 0 & 0 & J^{-T} & 0 & 0 & 0 & 0\\
0 & 0 & 0 & 0 & 0 & 0 & 0 & 0 & \Theta^{-1} & 0 & 0 & 0\\
0 & 0 & 0 & 0 & 0 & 0 & 0 & 0 & 0 & B^{1} & 0 & 0\\
0 & 0 & 0 & 0 & 0 & 0 & 0 & 0 & 0 & 0 & B^{2} & 0\\
0 & 0 & 0 & 0 & 0 & 0 & 0 & 0 & 0 & 0 & 0 & B^{3}\\
\end{bmatrix}
\end{equation}
որտեղ՝
\begin{equation}
J = \dfrac{\partial \left(p_{1}, p_{2}\right)}{\partial \left(x, y\right)}
\end{equation}
\begin{equation}
\Theta = \begin{bmatrix}
						\left(\dfrac{\partial p_{1}}{\partial x}\right)^{2} & 2\dfrac{\partial p_{1}}{\partial x}\dfrac{\partial p_{2}}{\partial x} & \left(\dfrac{\partial p_{2}}{\partial x}\right)^{2}\\
						\dfrac{\partial p_{1}}{\partial y}\dfrac{\partial p_{1}}{\partial x} & \dfrac{\partial p_{1}}{\partial y}\dfrac{\partial p_{2}}{\partial x} + \dfrac{\partial p_{1}}{\partial x}\dfrac{\partial p_{2}}{\partial y} & \dfrac{\partial p_{2}}{\partial y}\dfrac{\partial p_{2}}{\partial y}\\
						\left(\dfrac{\partial p_{1}}{\partial y}\right)^{2} & 2\dfrac{\partial p_{1}}{\partial y}\dfrac{\partial p_{2}}{\partial y} & \left(\dfrac{\partial p_{2}}{\partial y}\right)^{2}\\
		\end{bmatrix}
\end{equation}
\begin{equation}
\begin{aligned}
&B^{i} = \hat{G_{i}}J^{-T}G_{i}^{T} \\
&\hat{G_{i}} = \begin{bmatrix}
			\hat{n_{i}} & \hat{t_{i}}
		     \end{bmatrix}^{T}\\
&G_{i} = \begin{bmatrix}
			n_{i} & t_{i}
	    \end{bmatrix}^{T}
\end{aligned}
\end{equation}

$\hat{n}_{1}, \hat{n}_{2}, \hat{n}_{3}$-ը ստանդարտ եռանկյան կողերի միջնակետերի արտաքին նորմալներն են, իսկ $\hat{t}_{1}, \hat{t}_{2}, \hat{t}_{3}$-ը՝ ուղղորդ վեկտորները։ Ավելի կոնկրետ՝
\begin{equation}
\begin{aligned}
				&\hat{n}_{1}=\left[\dfrac{\sqrt{2}}{2}, \dfrac{\sqrt{2}}{2}\right], \; \hat{n}_{2}=\left[-1, 0\right], \; \hat{n}_{3}=\left[0, -1\right] \\
				&\hat{t}_{1}=\left[-\dfrac{\sqrt{2}}{2}, \dfrac{\sqrt{2}}{2}\right], \; \hat{t}_{2}=\left[0, -1\right], \; \hat{t}_{3}=\left[1, 0\right]
\end{aligned}
\end{equation}
$n_{i}$-ն և $t_{i}$-ն արտապարկերված եռանկյան համապատասխան միջնակետերի նորմալներն ու ուղղորդ վեկտորներն են (տես նկ. 2.11., 2.12)։
\begin{figure}[H]
\centering
\includegraphics[width=0.5\textwidth]{images/standard_triangle_vecs}
\captionsetup{labelformat=empty}
\caption{Նկար 2.12. Նորմալ և ուղղորդ վեկտորները ստանդարտ եռանկյան համար։}
\end{figure}
\begin{figure}[H]
\centering
\includegraphics[width=0.5\textwidth]{images/arb_triangle_vecs}
\captionsetup{labelformat=empty}
\caption{Նկար 2.13. Նորմալ և ուղղորդ վեկտորները կամայական եռանկյան համար։}
\end{figure}
\begin{equation}
D = \left[\begin{array}{cccccccccccc}1 & 0 & 0 & 0 & 0 & 0 & 0 & 0 & 0 & 0 & 0 & 0\\
0 & I_{2} & 0 & 0 & 0 & 0 & 0 & 0 & 0 & 0 & 0 & 0\\
0 & 0 & I_{3} & 0 & 0 & 0 & 0 & 0 & 0 & 0 & 0 & 0\\
0 & 0 & 0 & 1 & 0 & 0 & 0 & 0 & 0 & 0 & 0 & 0\\
0 & 0 & 0 & 0 & I_{2} & 0 & 0 & 0 & 0 & 0 & 0 & 0\\
0 & 0 & 0 & 0 & 0 & I_{3} & 0 & 0 & 0 & 0 & 0 & 0\\
0 & 0 & 0 & 0 & 0 & 0 & 1 & 0 & 0 & 0 & 0 & 0\\
0 & 0 & 0 & 0 & 0 & 0 & 0 & I_{2} & 0 & 0 & 0 & 0\\
0 & 0 & 0 & 0 & 0 & 0 & 0 & 0 & I_{3} & 0 & 0 & 0\\
0 & 0 & 0 & 0 & 0 & 0 & 0 & 0 & 0 & 1 & 0 & 0\\
0 & 0 & 0 & - \frac{15}{8 l_{1}} & - \frac{7 t_{1}}{16} & - \frac{\tau_{1} l_{1}}{32} & \frac{15}{8 l_{1}} & - \frac{7 t_{1}}{16} & \frac{\tau_{1} l_{1}}{32} & 0 & 0 & 0\\
0 & 0 & 0 & 0 & 0 & 0 & 0 & 0 & 0 & 0 & 1 & 0\\
\frac{15}{8 l_{2}} & - \frac{7 t_{2}}{16} & \frac{\tau_{2} l_{2}}{32} & 0 & 0 & 0 & - \frac{15}{8 l_{2}} & - \frac{7 t_{2}}{16} & - \frac{\tau_{2} l_{2}}{32} & 0 & 0 & 0\\
0 & 0 & 0 & 0 & 0 & 0 & 0 & 0 & 0 & 0 & 0 & 1\\- \frac{15}{8 l_{3}} & - \frac{7 t_{3}}{16} & - \frac{\tau_{3} l_{3}}{32} & \frac{15}{8 l_{3}} & - \frac{7 t_{3}}{16} & \frac{\tau_{3} l_{3}}{32} & 0 & 0 & 0 & 0 & 0 & 0
\end{array}\right]
\end{equation}
որտեղ $l_{j}$-ն $j$-րդ տողի երկարությունն է, իսկ $\tau_{j}$-ն սահմանվում է հետևյալ կերպ.
\begin{equation}
				\tau_{i}=\left[\left(t_{i}^{x}\right)^{2},  2t_{i}^{x}t_{i}^{y},  \left(t_{i}^{x}\right)^{2} \right]
\end{equation}
Նշանակենք նաև
\begin{equation}
E = E_{ij} = \begin{cases}
1, 1\leq i=j\leq19 \; \text{կամ} \; (i,j) \in \left\{(20, 21), (21, 23)\right\}, \\
0, \; \text{հակառակ դեպքում}
\end{cases}
\end{equation}
\begin{equation}
\begin{aligned}
			&M = \left(EVD\right)^{T}\\
			&F(f) = f \circ \left(\begin{bmatrix}
							\xi_{23} & \xi_{31} \\
							\eta_{23} & \eta_{31} \\
				\end{bmatrix}\begin{bmatrix}
							x \\
							y \\
							\end{bmatrix} + \begin{bmatrix}
							x_{1} \\
							y_{1} \\
							\end{bmatrix}\right)
\end{aligned}
\end{equation}
Այժմ, ունենալով ստանդարտ եռանկյան $\hat{\Psi}$ բազիսները, կարող ենք ստանալ արտապատկերված եռանկյան բազիսները հետևյալ բանաձևով.
\begin{equation}
\Psi = MF\left(\hat{\Psi}\right)
\end{equation} 
Ստորև ներկայացված են կառուցված բազիսային ֆունկցիաներ չորս եռանկյուն էլեմենտների վրա $U, \dfrac{\partial U}{\partial x}, \dfrac{\partial U}{\partial y}, \dfrac{\partial^{2} U}{\partial x^{2}}, \dfrac{\partial^{2} U}{\partial x\partial y}, \dfrac{\partial^{2} U}{\partial y^{2}}$ հերթականությամբ։
\begin{figure}[H]
  \centering
  \begin{minipage}[b]{0.4\textwidth}
    \includegraphics[width=\textwidth]{images/argyris_basis_func_value}
  \end{minipage}
  \hfill
  \begin{minipage}[b]{0.4\textwidth}
    \includegraphics[width=\textwidth]{images/argyris_basis_fx_value}
  \end{minipage}
\\
  \begin{minipage}[b]{0.4\textwidth}
    \includegraphics[width=\textwidth]{images/argyris_basis_fy_value}
  \end{minipage}
\hfill
  \begin{minipage}[b]{0.4\textwidth}
    \includegraphics[width=\textwidth]{images/argyris_basis_fxx_value}
  \end{minipage}
\\
  \begin{minipage}[b]{0.4\textwidth}
    \includegraphics[width=\textwidth]{images/argyris_basis_fxy_value}
  \end{minipage}
\hfill
  \begin{minipage}[b]{0.4\textwidth}
    \includegraphics[width=\textwidth]{images/argyris_basis_fyy_value}
  \end{minipage}
\\
  \begin{minipage}[b]{0.4\textwidth}
    \includegraphics[width=\textwidth]{images/argyris_basis_fn_value}
  \end{minipage}
\captionsetup{labelformat=empty}
\caption{Նկար 2.14. Արգիրիսի բազիսային ֆունկցիաները կառուցված չորս եռանկյունների վրա։}
\end{figure}

\newpage
\section*{\centering {\addfontfeatures{FakeBold=2.0}Գլուխ 3 \\ Եռաչափ մոտարկում}}
\setcounter{equation}{0}

Այժմ դիտարկենք երեք փոփոխականի ֆունկցիայի մոտարկման խնդիրը: Ինչպես նախորդ գլխում, այս դեպքում տիրույթի տրոհումը կարելի է իրականացնել կամայական ձևով։

\subsection*{{\addfontfeatures{FakeBold=2.0}3.1 Ուղղանկյունանիստ տիրույթ}}

Դիցուք տրված են $f:\Omega\mapsto \Theta$,  $\Theta \subset \mathbb{R}$, $\Omega \subset \mathbb{R}^{3} = \left[x_{0}, x_{M}\right] \times \left[y_{0}, y_{N}\right] \times \left[z_{0}, z_{K}\right]$  ուղղանկյունանիստ տիրույթը, որը տրոհված է $\left[x_{i}, x_{i+1}\right] \times \left[y_{j}, y_{j+1}\right] \times \left[z_{k}, z_{k+1}\right]$, ուղղանկյունանիստ էլեմենտների:
$$x_{i+1}-x_{i}=h_{1}, \; y_{j+1}-y_{j}=h_{2}, \;z_{k+1}-z_{k}=h_{3}, \; i=\overline{0, M-1}, j=\overline{0, N-1},  =\overline{0, K-1}$$

\subsubsection*{ {\addfontfeatures{FakeBold=2.0}3.1.1. Լագրանժի մոտարկում}}
Լագրանժի մոտարկման բանաձևերը ստանալու համար հարմար է տրված ուղղանկյունը արտապատկերել միավոր խորանարդի վրա, ինչպես արել էինք երկչափ դեպքում։ Միավոր խորանարդի կոորդինատական համակարգը նշանակենք $(p, q, r)$-ով։
Ստորև ներկայացված է գագաթների համարակալման հաջորդականությունը.
\begin{figure}[H]
\centering
\includegraphics[width=0.7\textwidth]{images/unit_cube}
\captionsetup{labelformat=empty}
\caption{Նկար 3.1. Գագաթների համարակալման հաջորդականությունը միավոր խորանարդի վրա։}
\end{figure}
Միավոր խորանարդի վրա բազիսային ֆունկցիաները տրվում են հետևյալ բանաձևերով։
\begin{equation}
\begin{aligned}
&\varphi^{(1)}_{1} = pqr \\
&\varphi^{(1)}_{2} = (1-p)qr \\
&\varphi^{(1)}_{3} = (1-p)(1-q)r \\
&\varphi^{(1)}_{4} = p(1-q)r \\
&\varphi^{(1)}_{5} = pq(1-r) \\
&\varphi^{(1)}_{6} = (1-p)q(1-r) \\
&\varphi^{(1)}_{7} = (1-p)(1-q)(1-r) \\
&\varphi^{(1)}_{8} = p(1-q)(1-r)
\end{aligned}
\end{equation}
Տրված $\left\{x_{i}, y_{i}, z_{i}\right\}$ ուղղանկյունանիստից դեպի միավոր խորանարդ արտապատկերման բանաձևերը հետևյալներն են.
\begin{equation}
\begin{aligned}
				&x=\sum_{i=1}^{8}\varphi_{i}\left(p, q, r\right)x_{i} \\
				&y=\sum_{i=1}^{8}\varphi_{i}\left(p, q, r\right)y_{i} \\
				&z=\sum_{i=1}^{8}\varphi_{i}\left(p, q, r\right)z_{i}
\end{aligned}
\end{equation}
\subsubsection*{ {\addfontfeatures{FakeBold=2.0}3.1.2. Էրմիթյան մոտարկում}}

{\addfontfeatures{FakeBold=2.0}{Եռգծային մոտարկում}}
Որպես բազիսային ֆունկցիա վերցնենք $(1.3)$-ում տրված ֆունկցիաների թենզորական արտադրյալները։
\begin{equation}
			 \varphi^{(ijk)}\left(x,y,z\right)=\varphi^{(i)}(x)\varphi^{(j)}(y)\varphi^{(k)}(z)
\end{equation}
Ավելի բարձր կարգի էրմիթյան մոտարկում ստանալու համար անհրաժեշտ է ստանալ համապատասխան բազիսային ֆունկցիաների թենզորական արտադրյալները (տես գլուխ 2)։
\newpage
\subsection*{{\addfontfeatures{FakeBold=2.0}3.2 Քառանիստներով մոտարկվող տիրույթ}}

Դիցուք տրված են $f:\Omega\mapsto \Theta, \Theta \subset \mathbb{R}, \Omega \subset \mathbb{R}^{3} $ եռաչափ տիրույթը, որը կամայական ձևով տրոհված է քառանկյուն էլեմենտների։


\subsubsection*{{\addfontfeatures{FakeBold=2.0}3.2.1. Լագրանժի մոտարկում}}

Յուրաքանչուր քառանիստ էլեմենտի վրա դիտարկենք $m$-րդ կարգի լրիվ բազմանդամ
\begin{equation}
				F_{m}\left(x,y,z\right)=\sum_{i+j+k=0}^{m}\alpha_{ijk}x^{i}y^{j}z^{k}
\end{equation}

որը կարող է օգտագործվել որպես մոտարկող ֆունկցիա $\dfrac{1}{6}\left(m+1\right)\left(m+2\right)\left(m+3\right)$ սիմետրիկ դասավորված կետերի վրա։

Յուրաքանչյուր $P_{1}, P_{2}, P_{3}, P_{4}$ գագաթներով քառանկյուն էլեմենտի համար դիտարկենք հետևյալ մոտարկող ֆունկցիան.
\begin{equation}
F_{1}(x, y, z) = \sum_{i=1}^{4} p^{(1)}_{i}(x,y,z)f(x_{i}, y_{i}, z_{i})
\end{equation}
$p^{(1)}_{i}$ ֆունկցիաները սահմանվում են հետևյալ կերպ.
\begin{equation}
p^{(1)}_{i}(x,y) = \dfrac{1}{\Gamma_{ijkl}}\left(E_{jkl} - A_{jkl}x + B_{jkl}y - C_{jkl}z\right)
\end{equation}
որտեղ
\begin{equation}
				\Gamma_{ijkl} = \begin{vmatrix}
										 1 & x_{i} & y_{i} & z_{i} \\
										 1 & x_{j} & y_{j} & z_{j} \\
										 1 & x_{k} & y_{k} & z_{k} \\
										 1 & x_{l} & y_{l} & z_{l}
							\end{vmatrix}
\end{equation}
\begin{equation}
				E_{jkl} = \begin{vmatrix}
										 x_{i} & y_{j} & z_{j} \\
										 x_{j} & y_{k} & z_{k} \\
										 x_{k} & y_{l} & z_{l}
							\end{vmatrix},
				A_{jkl} = \begin{vmatrix}
										 1 & y_{j} & z_{j} \\
										 1 & y_{k} & z_{k} \\
										 1 & y_{l} & z_{l}
							\end{vmatrix},
				B_{jkl} = \begin{vmatrix}
										 1 & x_{j} & z_{j} \\
										 1 & x_{k} & z_{k} \\
										 1 & x_{l} & z_{l}
							\end{vmatrix},
				C_{jkl} = \begin{vmatrix}
										 1 & x_{j} & y_{j} \\
										 1 & x_{k} & y_{k} \\
										 1 & x_{l} & y_{l}
							\end{vmatrix}
\end{equation}

որտեղ $(x_{i}, y_{i}, z_{i}), \; i=1, 2, 3, 4$ տրված քառանկյուն էլեմենտի գագաթներն են։

Երկչափ մոտարկման նմանությամբ, այստեղ ևս $(3.6)$ բանաձևը ոչ այլ ինչ է, քան կամայական քառանիստի արտապատկերում դեպի ստանդարտ քառանկյան գագաթների $(1, 0, 0), (0, 1, 0), (0, 0, 0), (0, 0, 1)$ հերթականությամբ (տես նկ. 3.2)։ Հակադարձ ձևափոխությունը տրվում է հետևյալ բանաձևերով.
\begin{equation}
\begin{aligned}
				&x = x_{3} + x \left(- x_{3} + x_{4}\right) + y \left(x_{1} - x_{3}\right) + z \left(x_{2} - x_{3}\right) \\
				&y = y_{3} + x \left(- y_{3} + y_{4}\right) + y \left(y_{1} - y_{3}\right) + z \left(y_{2} - y_{3}\right) \\
				&z = z_{3} + x \left(- z_{3} + z_{4}\right) + y \left(z_{1} - z_{3}\right) + z \left(z_{2} - z_{3}\right)
\end{aligned}
\end{equation}
\begin{figure}[H]
\centering
\includegraphics[width=0.7\textwidth]{images/tetrahedron}
\captionsetup{labelformat=empty}
\caption{Նկար 3.2. Ստանդարտ քառանկյուն։}
\end{figure}


Քառակուսային և խորանարդային մոտարկումները և դրանց բազիսային ֆունկցիաները կառուցվում են երկչափ մոտարկման նմանությամբ։
\newpage
\section*{\centering {\addfontfeatures{FakeBold=2.0}Գլուխ 4 \\ Վարիացիոն մեթոդ}}
\setcounter{equation}{0}

Մաթեմատիկական ֆիզիկայի բազմաթիվ խնդիրներ կապված են վարիոցիոն մեթոդի հետ, որը մաթեմատիկական անալիզի գլխավոր մեթոդներից է։

\subsection*{Խնդրի դրվածքը}
\hspace{\parindent}Վարիացիոն մեթոդի սկզբունքն այն է, որ դիտարկվող ֆունկցիայի ինտեգրալը տրված տիրույթում ընդունում է մեծագույն (փոքրագույն) արժեք տվյալ համակարգի իրական վիճակի համար, համեմատած բոլոր հնարավոր վիճակների բազմության հետ։  Ենթաինտեգրալային ֆունկցիան կախված է տրված կոորդինատներից, ֆունկցիայի արժեքից, նրա ածանցյալներից, իսկ ինտեգրումը կատարվում է տրված կոորդինատական համակարգում, որը կարող է ներառել նաև ժամանակը։ Մինիմումի որոշման խնդիրը հաճախ բերվում է մի քանի դիֆերենցիալ հավասարումների, համապատասխան եզրային պայմաններով։  Այն իրական փոփոխականի ֆունկցիայի էքստրեմումի որոնման խնդրի ընդհանրացումն է, որտեղ տրված ֆունկցիայի համար կոմպակտ տիրույթում անհրաժեշտ է գտնել այպիսի կետեր, որոնք մինիմում (մաքսիմում) են իրենց որևէ շրջակայքում։
Վարիացիոն մեթոդում ֆունկցիոնալը ինտեգրալ է, որը կախված է ֆունկցիայից, որի որոշման տիրույթը թույլատրելի ֆունկցիաների  տարածությունն է։
Այս  մեթոդի հիմնական դժվարությունը կայանում է նրանում, որ խնդիրները, որոնք կարող են ձևակերպվել որպես վարիացիոն, հնարավոր է, որ լուծում չունենան այն պատճառով, որ ֆունկցիոնալ տարածություները  կոմպակտ չեն։
Սակայն վարիացիոն մեթոդի հիմնական առավելությունն այն է, որ դրա կիրառման համար դրվող պահանջները ավելի թույլ են:
\newpage
\subsection*{{\addfontfeatures{FakeBold=2.0}Օրինակներ}}
\hspace{\parindent}Որպես օրինակ դիտարկենք հետևյալ կրկնակի ինտեգրալը.
\begin{equation}
I\left(f\right)=\iint \limits_{\Omega} F\left(x, y, f, f_{x}, f_{y}\right)dxdy
\end{equation}
որտեղ $f \in C^{2}(\Omega)$,  $\Omega\subset \mathbb{R}^{2}$ և որի արժեքները որոշված են $\partial \Omega$-ում:
Այս դեպքում, որպեսզի $f$ ֆունկցիան $\left(1\right)$ ֆունկցիոնալի համար հանդիսանա մինիմում, անհրաժեշտ է, որ $f(x,y)$ ֆունկցիան բավարարի Էյլեր-Լագրանժի հավասարմանը, հավելելով համապատասխան եզրային պայմանները։
\begin{equation}
\dfrac{\partial}{\partial x}F_{u_{x}} + \dfrac{\partial}{\partial y}F_{u_{y}} - F_{u} = 0
\end{equation}
Օրինակ, $F = \dfrac{1}{2}\left(u_{x}^2+u_{y}^2\right)$ դեպքում խնդիրը բերվում է Լապլասի հավասարմանը.
\begin{equation}
\dfrac{\partial^{2}f}{\partial x^{2}} + \dfrac{\partial^{2}f}{\partial y^{2}} = 0
\end{equation}
Այժմ պարզ է, որ երկրորդ կարգի ածանցյալների անընդհատությունը անհրաժեշտ է Էյլեր-Լագրանժի հավասարման գոյության համար։ Բայց վարիացիոն մեթոդը պահանջում է միայն $f$-ի անընդհատություն և առաջին կարգի մասնակի ածանցյալների կտոր առ կտոր անընդհատություն։

Դիֆերենցիալ հավասարումը, որը կապված է վարիացիոն խնդրի հետ, կոչվում է Էյլեր-Լագրանժի հավասարում։ Այն միայն անհրաժեշտ պայման է, որին պետք է բավարարի ֆունկցիան, որը մինիմիզացնում (մաքսիմիզացնում) է ֆունկցիոնալը։

\newpage
\noindent Ստորև դիտարկվում են խնդիրների մի քանի տարբերակներ.
%This command allows to indent only the first line of enumeration
\setlist[enumerate]{itemindent=\dimexpr\labelwidth+\labelsep\relax,leftmargin=0pt}

\begin{enumerate}
\item Մեկ փոփոխականի ֆունկցիա.
$$f \in \Omega \mapsto \Theta, \; \Omega, \Theta \subset \mathbb{R}$$
\noindent Այս դեպքում ֆունկցիոնալն ընդունում է հետևյալ տեսքը.
$$I\left(f\right) = \int\limits_{x_0}^{x_{1}} F\left(x, f, f^{'}\right)dx$$
\noindent որտեղ $f(x_{0})$ և $f(x_{1})$, $x_{0}, x_{1} \in \Theta$ ը տրված են։
\noindent Մինիմումի գոյության անհրաժեշտ պայման է հետևյալ դիֆերենցիալ հավասարումը.
$$\dfrac{\partial F}{\partial f} - \dfrac{d}{dx}\dfrac{\partial F}{\partial f^{'}}=0$$
Որը համարժեք է.
$$\dfrac{d^{2}f}{dx^{2}}F_{f^{'}f^{'}}+\dfrac{df}{dx}F_{f^{'}f} + F_{f^{'}x}-F_{f}=0$$
\item Մի քանի անհայտ ֆունկցիաներ.
$$f,g \in \Omega \mapsto \Theta, \; \Omega, \Theta \subset \mathbb{R}$$
\noindent Այս դեպքում ֆունկցիոնալն ընդունում է հետևյալ տեսքը.
$$I\left(f\right) = \int\limits_{x_0}^{x_{1}} F\left(x, f, g, f^{'}, g^{'}\right)dx$$
\noindent որտեղ $f(x_{0}), \; f(x_{1}), \; g(x_{0}), \; g(x_{1})$, $x_{0}, x_{1} \in \Theta$ ը տրված են։
\noindent Մինիմումի գոյության անհրաժեշտ պայման է հետևյալ դիֆերենցիալ հավասարումների համակարգը.
$$\dfrac{\partial F}{\partial f} - \dfrac{d}{dx}\dfrac{\partial F}{\partial f^{'}}=0$$
$$\dfrac{\partial F}{\partial g} - \dfrac{d}{dx}\dfrac{\partial F}{\partial g^{'}}=0$$
\item Բարձր կարգի ածանցյալներ.
$$f \in \Omega \mapsto \Theta, \; \Omega, \Theta \subset \mathbb{R}$$

\noindent Այս դեպքում ֆունկցիոնալն ընդունում է հետևյալ տեսքը.
$$I\left(f\right) = \int\limits_{x_0}^{x_{1}} F\left(x, f, f^{'}, f^{''}, \dots, f^{(n)}\right)dx$$

\noindent որտեղ $f(x_{0}), \; f(x_{1}), \; f^{'}(x_{0}), \; f^{'}(x_{1})$, $x_{0}, x_{1} \in \Theta$ ը տրված են։
\noindent Մինիմումի գոյության անհրաժեշտ պայման է հետևյալ դիֆերենցիալ հավասարումը.

$$\dfrac{\partial F}{\partial f} - \dfrac{d}{dx}\dfrac{\partial F}{\partial f^{'}} + \dfrac{d^{2}}{dx^{2}}\dfrac{\partial F}{\partial f^{''}} - \dots (-1)^{n}\dfrac{d^{n}}{dx^{n}}\dfrac{\partial F}{\partial f^{(n)}}=0$$

\item Մի քանի անկախ փոփոխականներ.
$$f \in \Omega \mapsto \Theta, \; \Omega \subset \mathbb{R}^{n}, \Theta \subset \mathbb{R}$$

\noindent Այս դեպքում ֆունկցիոնալն ընդունում է հետևյալ տեսքը.
$$I\left(f\right) = \idotsint \limits_{\Omega} F\left(x, f, f_{x_{1}}, f_{x_{2}}, \dots f_{x_{n} }\right)d\Omega$$
\noindent որտեղ $f$ ֆունկցիայի արժեքները $\partial \Omega$-ի վրա տրված են։

\noindent Մինիմումի  գոյության անհրաժեշտ պայման է հետևյալ դիֆերենցիալ հավասարումը.

$$\dfrac{\partial F}{\partial f} - \dfrac{d}{dx_{1}}\dfrac{\partial F}{\partial f_{x_{1}}} - \dots -\dfrac{d}{dx_{n}}\dfrac{\partial F}{\partial f_{x_{n}}}=0$$
\end{enumerate}
\newpage
\subsubsection*{{\addfontfeatures{FakeBold=2.0}Եզրային պայմաններ}}
Նախորդիվ դիտարկել էինք այնպիսի խնդիրներ, որտեղ ֆունկցիայի արժեքները տրված տիրույթի եզրի տրված են։Սակայն որոշ խնդիրնորում ֆունկցիան տիրույթի եզրում տրված չէ, և տրվում են այլ տիպի եզրային պայմաններ։Նման դեպքերում որպես եզրային պայման դիտարկվում են հետևյալ պայմանները.

\noindent Մեկ փոփոխականի ֆունկցիա
$$\dfrac{\partial F}{\partial f^{'}}=0$$
Երկու անհայտ ֆունկցիա
$$\dfrac{\partial F}{\partial f^{'}}=\dfrac{\partial F}{\partial g^{'}}$$
Երկու փոփոխականի ֆունկցիա
$$F_{u_{x}}\dfrac{dy}{ds}-F_{u_{y}}\dfrac{dx}{ds}=0$$

\noindent որոնք հայտնի են որպես \emph{բնական կամ գլխավոր}  եզրային պայմաններ։

\subsubsection*{{\addfontfeatures{FakeBold=2.0}Պայմանական էքստրեմում}}
Այսպիսի վարիացիոն խնդիրներում անհրաշետ է մինիմիզացնել (մաքսիմիզացնել) տրված ֆունկցիոնալը, պայմանով, որ մեկ այլ ֆունկցիոնալ ընդունում է որևէ ֆիքսված արժեք այդ ֆունկցիայի համար։ Այսինքն, եթե դիտարկենք ֆունկցիանալ, որի արժեքների բազմությունը մեկ փոփոխականի ֆունկցիաներ են, ապա կունենանք.

$$I\left(f\right)=\int \limits_{x_{0}}^{x_{1}} F\left(x, f, f^{'}\right)dx $$
$$\int \limits_{x_{0}}^{x_{1}} G\left(x, f, f^{'}\right)dx = C$$

Այս խնդրի համար Էյլեր-Լագրանժի հավասարումը հետևյալն է.

$$\dfrac{\partial \left(F+\lambda G\right)}{\partial f} - \dfrac{d}{dx} \dfrac{\partial \left(F+\lambda G\right)}{\partial f^{'}} = 0$$


\newpage

%TODO: add description of finite elements

\section*{\centering {\addfontfeatures{FakeBold=2.0}Գլուխ 5 \\ Մոտավոր մեթոդներ։ Վերջավոր էլեմենտների մեթոդ}}
\setcounter{equation}{0}
Խնդիրների վարիացիոն մեթոդով ձևակերպումը, և վարիացիոն մեթոդների ավելի թույլ պայմանները թույլ են տալիս այդ խնդիրները լուծել մոտավոր մեթոդներով, որոնք հաճախ անվանվում են ուղիղ մեթոդներ։ Ուղիղ մեթոդներից է Ռիտցի մինիմիզացնող հաջորդականության մեթոդը, որը քննության կառնենք։ 

\subsection*{{\addfontfeatures{FakeBold=2.0}Ռիտցի մինիմիզացնող հաջորդականության մեթոդ}}
Դիտարկենք որևէ վարիացիոն մեթոդով տրված մինիմիզացիայի խնդիր.
		$$I\left(f\right) \longrightarrow min, \; f \in \Gamma$$
որտեղ $I$ ֆունկցիոնալը տրված տիրույթում որոշյալ ինտեգրալ է։
Մոտավոր լուծում կարելի է ստանալ, եթե ֆունկցիոնալի արժեքների բազմությունը սահմանափակենք որևէ վերջավոր չափանի ենթատարածությունով, որն ունի $\{\varphi_{j}\}_{j=1}^{N}$ բազիս։
			  $$\Gamma_{N} \in \Gamma $$
Ենթադրենք, որ $I$ ֆունկցիոնալի արժեքների տիրույթը ունի ճշգրիտ ստորին եզր, նշանակենք այն $\alpha_{0}$ ով։
Այդ դեպքում գոյություն ունի $\{f_{j}\}_{j=1}^{\infty}$ հաջորդականություն այնպիսին, որ
\begin{equation}
\lim_{n \to \infty}I\left(f_{n}\right) = \lim_{n \to \infty}I\left( \sum_{j=1}^{n} \gamma_{j}\varphi_{j} \right) = \alpha_{0}
\end{equation}
և ֆունկցիոնալի որոշման տիրույթի ցանկացած այլ $g$ ֆունկցիայի համար
			$$I\left(g\right) \geq \alpha_{0}$$
լուծումը փնտրենք հետևյալ կերպ.
\begin{equation}
f_{0}=\sum_{j=1}^{N}\gamma_{j}\varphi_{j}, \; \gamma_{j} \in \mathbb{R}
\end{equation}
Այդ դեպքում խնդիրը բերվում է սովորական մինիմումի խնդրի.
\begin{equation}
\dfrac{\partial I}{\partial \gamma_{i}} = \dfrac{\partial}{\partial \gamma_{i}} I \left(\sum_{j=1}^{N}\gamma_{j}\varphi_{j}\right) = 0
\end{equation}
\newpage
\subsection*{{\addfontfeatures{FakeBold=2.0}5.1 Պուասոնի հավասարման լուծում ուղղանկյուն տիրույթում}}

Դիտարկենք հետևյալ դիֆերենցիալ հավասարումը.
\begin{equation}
\begin{cases}
			\Delta u =f \\
			u \Big |_{\partial D} = 0
\end{cases}
\end{equation}
որտեղ $D = \left[x_{0}, x_{N}\right] \times \left[y_{0}, y_{M}\right]$:

Այս դիֆերենցիալ հավասարման  համապատասխան վարիացիոն խնդիրը կլինի.
\begin{equation}
I(u) = \frac{1}{2}\iint \limits_{D} \left[u_x^2 + u_y^2 \right]dxdy + \iint \limits_{D} fudxdy \longrightarrow min
\end{equation}
D տիրույթը տրոհենք ուղղանկյուն եղանկյունների, և որպես բազիսային ֆունկցիաներ վերցնենք Կուրանտի ֆունկցիաները $\left(2.6\right)$։
Որոնելի ֆունկցիան կփնտրենք բազիսային ֆունկցիաների գծային կոմբինացիայի տեսքով.
\begin{equation}
u(x,y) = \sum_{i=0}^{N} \sum_{j=0}^{M} u_{ij}\varphi^{(ij)}(x,y)
\end{equation}
Ուստի ինտեգրալային ֆունկցիոնալը կունենա հետևյալ տեսքը.
\begin{equation}
$$\scalemath{0.9}{\frac{1}{2} \iint \limits_{D}\left[\left(\sum_{i=0}^{N} \sum_{j=0}^{M}u_{ij}\varphi_{x}^{(ij)}(x,y)\right)^2 + \left(\sum_{i=0}^{N} \sum_{j=0}^{M}u_{ij}\varphi_{y}^{(ij)}(x,y)\right)^2+2f(x,y)\left( \sum_{i=0}^{N} \sum_{j=0}^{M}u_{ij}\varphi^{(ij)}(x,y)\right)\right]dxdy}
\end{equation}
Համաձայն էքստրեմումի անհրաժեշտ պայմանի.
$$\dfrac{\partial I}{ \partial u_{kl}} = \dfrac{\partial}{\partial u_{kl}} I \left(\sum_{j=0}^{M} u_{ij}\varphi^{(ij)}(x,y)\right) = 0 $$

Դիտարկենք առաջին կրկնակի գումարը.

$$\left(\sum_{i=0}^{N} \sum_{j=0}^{M}u_{ij}\varphi_{x}^{(ij)}(x,y)\right)^2 = \left[u_{kl}\varphi_{x}^{(kl)}(x,y)\right]^{2} + 2u_{kl}\varphi_{x}^{(kl)}(x,y)[\dots] + [\dots]^2$$
որտեղ բազմակետերով արտահայտությունը իր մեջ չի պարունակում $u_{kl}$ ը։
Հանգունորեն, երկրորդ կրկնակի գումարի համար՝
$$\left(\sum_{i=0}^{N} \sum_{j=0}^{M}u_{ij}\varphi_{y}^{(ij)}(x,y)\right)^2 = \left[u_{kl}\varphi_{y}^{(kl)}(x,y)\right]^{2} + 2u_{kl}\varphi_{y}^{(kl)}(x,y)[\dots] + [\dots]^2$$

Այսպիսով ավելի պարզեցված տեսքով համակարգը ունի հետևյալ տեսքը.
$$\scalemath{0.75}{\frac{\partial}{\partial u_{kl}}I \left(u\right)= \iint \limits_{D} \left[2u_{kl}\left\{\varphi_{x}^{(kl)}(x,y)\right\}^{2} + 2\varphi_{x}^{(kl)}(x,y)[\dots] + 2u_{kl}\left\{\varphi_{y}^{(kl)}(x,y)\right\}^{2} + 2\varphi_{y}^{(kl)}(x,y)[\dots] + 2f(x,y)\varphi^{(kl)}(x,y)\right]dxdy}$$
Դիտարկենք $\varphi_{x}^{(kl)}(x,y) $ և $\varphi_{y}^{(kl)}(x,y) $ ֆունկցիաները։
\begin{equation}
\varphi_{x}^{(kl)}(x,y)  = \begin{cases}
\phantom{-}0, &(x,y) \in S_{1} \\
\phantom{-}h_{1}^{-1}, &(x,y) \in S_{2} \\
\phantom{-}h_{1}^{-1}, &(x,y) \in S_{3} \\
\phantom{-}0, &(x,y) \in S_{4} \\
-h_{1}^{-1}, &(x,y) \in S_{5} \\
-h_{1}^{-1}, &(x,y) \in S_{6}\\
\phantom{-}0, &\text{մնացած դեպքերում}
\end{cases}
\end{equation}
\begin{equation}
\varphi_{y}^{(kl)}(x,y)  = \begin{cases}
-h_{2}^{-1}, &(x,y) \in S_{1} \\
-h_{2}^{-1}, &(x,y) \in S_{2} \\
\phantom{-}0, &(x,y) \in S_{3} \\
\phantom{-}h_{2}^{-1}, &(x,y) \in S_{4} \\
\phantom{-}h_{2}^{-1}, &(x,y) \in S_{5} \\
\phantom{-}0, &(x,y) \in S_{6}\\
\phantom{-}0, &\text{մնացած դեպքերում}
\end{cases} \;
\end{equation}

Քանի որ $\varphi_{x}^{(kl)}(x,y) $ և $\varphi_{y}^{(kl)}(x,y) $ ֆունկցիաները ոչ զրոյական են $\left[x_{k-1, l}, x_{k+1, l} \right] \times \left[y_{k, l-1}, y_{k, l+1}\right]$-ում, ապա

\begin{equation}
\iint \limits_{D} 2u_{kl}\left\{\varphi_{x}^{(kl)}(x,y)\right\}^{2}dxdy = 4u_{kl}\dfrac{h_{2}}{h_{1}}
\end{equation}
\begin{equation}
\iint \limits_{D} 2u_{kl}\left\{\varphi_{y}^{(kl)}(x,y)\right\}^{2}dxdy = 4u_{kl}\dfrac{h_{1}}{h_{2}}
\end{equation}
Քանի որ յուրաքանչյուր $\varphi_{kl}$ բազիսային ֆունկցիա հատվում Է $\varphi_{k-1, l}$, $\varphi_{k+1, l}$, $\varphi_{k, l-1}$, $\varphi_{k, l+1}$ ֆունկցիաների հետ, ապա
\begin{equation}
\iint \limits_{D} 2\varphi_{x}^{(kl)}(x,y)[\dots]dxdy = 2\iint \limits_{D}\left[\varphi_{x}^{(kl)}(x,y)\sum_{\substack{r=k-1\\ r \neq k}}^{k+1} \sum_{\substack{s=l-1\\ s \neq l}}^{l+1} u_{rs} \varphi_{x}^{(rs)}\right]dxdy= -2u_{k-1, l}\dfrac{h_{2} }{h_{1}} -2u_{k+1, l}\dfrac{ h_{2}}{h_{1}}
\end{equation}
Հանգունորեն՝
\begin{equation}
\iint \limits_{D} 2\varphi_{y}^{(kl)}(x,y)[\dots]dxdy = 2\iint \limits_{D}\left[\varphi_{y}^{(kl)}(x,y)\sum_{\substack{r=k-1\\ r \neq k}}^{k+1} \sum_{\substack{s=l-1\\ s \neq l}}^{l+1} u_{rs} \varphi_{y}^{(rs)}\right]dxdy=-2u_{k, l-1}\dfrac{h_{1}} {h_{2}} - 2u_{k, l+1}\dfrac{h_{1}}{ h_{2}}
\end{equation}
Եվ վերջապես
\begin{equation}
2\iint \limits_{D} f(x,y)\varphi^{(kl)}(x,y)dxdy \approx 2 f_{kl} h_{1} h_{2}
\end{equation}
Դիրիխլեի եզրային պայմանների համար կունենանք.
$$u_{0, l} = u_{N, l} = u_{k, 0} = u_{k, M} = 0$$
Այսպիսով, ստացանք հետևյալ հավասարումների համակարգը.
\begin{equation}
\begin{cases}
			2\left[\dfrac{1}{h^{2}_{1}} + \dfrac{1}{h^{2}_{2}}\right] u_{kl}  -\dfrac{1}{h^{2}_{1}}u_{k-1, l} - \dfrac{1}{h^{2}_{1}}u_{k+1, l} -\dfrac{1}{h^{2}_{2}}u_{k, l-1} - \dfrac{1}{h^{2}_{2}}u_{k, l+1} + f_{kl} = 0\\
			u_{0, l} = u_{N, l} = u_{k, 0} = u_{k, M} = 0
\end{cases}
\end{equation}
\newpage
\subsubsection*{{\addfontfeatures{FakeBold=2.0}Ծրագրային իրականացում}}

Պուասոնի հավասարման մոտավուր լուծումը իրականացնելու համար օգտվենք Python ծրագրավորմալ լեզվից, օգտագործելով Numpy գրադարանը, որը հարմար է բազմաչափ զանգվածների հետ մաթեմատիկական գործողություններ իրականացնալու համար։ Վիզուալիզացիաները կառուցելու համար կօգտվենք Matplotlib գրադարանից։

Ծրագրի սկզբում տրվում է ուղղանկյուն տիրույթի սահմանները և դրա տրոհման $h_{1}$ և $h_{2}$ քայլերը, $f$ ֆունկցիան։ Հաջորդիվ կազմվում է հավասարումների համակարգը, կանչվում այն լուծող ֆունկցիան։ Այսնուհետև բազիսային ֆունկցիաների միջոցով կառուցվում է մոտարկող ֆունկցիան։

Օրինակ


				$$
					\begin{cases}
								\Delta u =2 \\
								u \Big |_{\partial D} = 0
					\end{cases}
				$$

				$$ D = \left[-\dfrac{\pi}{2}, \phantom{-}\dfrac{\pi}{2}\right] \times \left[-\dfrac{\pi}{2}, \phantom{-}\dfrac{\pi}{2}\right], \; h_{1}=h_{2}=\dfrac{\pi}{200}$$
Խնդրի համար ստացված լուծումը.
\begin{figure}[H]
\centering
\includegraphics[width=0.7\textwidth]{images/poisson_solution}
\captionsetup{labelformat=empty}
\caption{Նկար 5.1. Պուասոնի հավասարման լուծման գրաֆիկական ներկայացում։}
\end{figure}
\newpage
\subsection*{{\addfontfeatures{FakeBold=2.0}5.2 Պուասոնի հավասարման լուծում եռանկյունացվող տիրույթում}}

Դիտարկենք հետևյալ դիֆերենցիալ հավասարումը.
\begin{equation}
\begin{cases}
			\Delta u =f \\
			u \Big |_{\partial D} = 0
\end{cases}
\end{equation}
որտեղ $D$-ն կամայական կապակցված միակապ կոմպակտ տիրույթ է։

Այս դիֆերենցիալ հավասարման  համապատասխան վարիացիոն խնդիրը կլինի.
\begin{equation}
I(u) = \frac{1}{2}\iint \limits_{D} \left[u_x^2 + u_y^2 \right]dxdy + \iint \limits_{D} fudxdy \longrightarrow min
\end{equation}

D տիրույթը տրոհենք $N$ եռանկյունների։ Գագաթների համարակալումը կկատարենք հետևյալ կերպ. յուրաքանչյուր եռանկյան ներսում գագաթները կհամարակալենք $1, 2, 3$ թվերով ժամսլաքի հակառակ ուղղությամբ, որին կանվանենք լոկալ համարակալում, միևնույն ժամանակ գագաթները համարակալելով ամբողջ տիրույթի համար, որին կանվանենք գլոբալ համարակալում։ 
\begin{figure}[H]
\centering
\includegraphics[width=0.7\textwidth]{images/global_and_local_numbering}
\captionsetup{labelformat=empty}
\caption{Նկար 5.2. Գագաթների լոկալ (թավ տառատեսակ) և գլոբալ համարակալման օրինակ։}
\end{figure}
Յուրաքանչյուր $\Delta_{n}$ եռանկյան վրա ($n=\overline{1, N}$) որպես բազիսային ֆունկցիաներ վերցնենք $\left(2.18\right)$-ում ներկայացված ֆունկցիաները, և որոնելի ֆունկցիան փնտրենք բազիսային ֆունկցիաների գծային կոմբինացիայի տեսքով.
\begin{equation}
u(x, y) = \dfrac{1}{S_{n}}\sum_{i=1}^{3}u_{ni}p^{(ni)}(x,y)
\end{equation}
որտեղ $ni$-ն $n$-րդ եռանկյան $i$-րդ գագաթի համապատասխան ինդեքսն է, իսկ $p^{(ni)}(x,y)$-ն այդ գագաթի համապատասխան բազիսային ֆունկցիան է։
Ուստի ինտեգրալային ֆունկցիոնալը կունենա հետևյալ տեսքը.
\begin{equation}
$$\scalemath{0.95}{\frac{1}{2S_{n}^{2}} \iint \limits_{\Delta_{n}}\left[\left(\sum_{i=1}^{3}u_{ni}p^{(ni)}_{x}(x,y)\right)^2 + \left(\sum_{i=1}^{3}u_{ni}p^{(ni)}_{y}(x,y)\right)^2 +2S_{n}f(x,y)\left(\sum_{i=1}^{3}u_{ni}p^{(ni)}(x,y)\right)\right]dxdy}
\end{equation}
Համաձայն էքստրեմումի անհրաժեշտ պայմանի.
\begin{equation}
\dfrac{\partial I}{ \partial u_{nk}} = \dfrac{\partial}{\partial u_{nk}} I \left(\dfrac{1}{S_{n}}\sum_{i=1}^{3}u_{ni}p^{(ni)}(x,y)\right) = 0
\end{equation}
Դիտարկենք $\left(4.18\right)$ ինտեգրալի ենթաինտեգրալային արտահայտության ձախ մասի երկու մասերը.
\begin{equation}
\begin{aligned}
&\left(\sum_{i=1}^{3}u_{ni}p^{(ni)}_{x}(x,y)\right)^2=\left(u_{nk}p_{x}^{(nk)}\right)^{2}+2u_{nk}u_{nl}p_{x}^{(nk)}p_{x}^{(nl)}+2u_{nk}u_{nm}p_{x}^{(nk)}p_{x}^{(nm)} \\
&\left(\sum_{i=1}^{3}u_{ni}p^{(ni)}_{y}(x,y)\right)^2=\left(u_{nk}p_{y}^{(nk)}\right)^{2}+2u_{nk}u_{nl}p_{y}^{(nk)}p_{y}^{(nl)}+2u_{nk}u_{nm}p_{y}^{(nk)}p_{y}^{(nm)} \\
\end{aligned}
\end{equation}

\noindent որտեղ $k, l, m$-ը գագաթների հաջորդականությունն է լոկալ համարակալման մեջ։

Այսպիսով ավելի պարզեցված տեսքով համակարգը ունի հետևյալ տեսքը.
\begin{equation}
\scalemath{0.75}{\iint \limits_{\Delta_{n}} \left[u_{nk}\left(\left(p_{x}^{(nk)}\right)^{2}+\left(p_{y}^{(nk)}\right)^{2}\right)+u_{nl}\left(p_{x}^{(nk)}p_{x}^{(nl)}+p_{y}^{(nk)}p_{y}^{(nl)}\right)+u_{nm}\left(p_{x}^{(nk)}p_{x}^{(nm)}+p_{y}^{(nk)}p_{y}^{(nm)}\right) + S_{n}u_{nk}p^{(nk)}\right]dxdy=0}
\end{equation}
Քանի որ
$$p_{x}^{(nk)} = \left(y_{nl}-y_{nm}\right)$$
$$p_{y}^{(nk)} = -\left(x_{nl}-x_{nm}\right)$$
ապա
\begin{equation}
\begin{aligned}
&\iint \limits_{\Delta_{n}}2u_{nk}\left(p_{x}^{(nk)}\right)^{2}dxdy = 2u_{nk} S_{n} \left(y_{nl}-y_{nm}\right)^{2} \\
&\iint \limits_{\Delta_{n}}2u_{nl}p_{x}^{(nk)}p_{x}^{(l)}dxdy = 2u_{nl} S_{n} \left(y_{nl}-y_{nm}\right)\left(y_{nm}-y_{nk}\right) \\
&\iint \limits_{\Delta_{n}}2u_{nm}p_{x}^{(nk)}p_{x}^{(nm)}dxdy = 2u_{nm} S_{n} \left(y_{nl}-y_{nm}\right)\left(y_{nk}-y_{nl}\right) \\
&\iint \limits_{\Delta_{n}}2u_{nk}\left(p_{y}^{(nk)}\right)^{2}dxdy = 2u_{nk} S_{n} \left(x_{nl}-x_{nm}\right)^{2} \\
&\iint \limits_{\Delta_{n}}2u_{nl}p_{y}^{(nk)}p_{y}^{(nl)}dxdy = 2u_{nl} S_{n} \left(x_{nl}-x_{nm}\right)\left(x_{nm}-x_{nk}\right) \\
&\iint \limits_{\Delta_{n}}2u_{nm}p_{y}^{(nk)}p_{y}^{(nm)}dxdy = 2u_{nm} S_{n} \left(x_{nl}-x_{nm}\right)\left(x_{nk}-x_{nl}\right) \\
&\iint \limits_{\Delta_{n}}2S_{n}f\left(x,y\right)p^{(nk)}dxdy \approx \dfrac{2}{3}u_{nk} S_{n}^{2}f_{nk} \\
\end{aligned}
\end{equation}
Նշանակենք
$$\begin{aligned}
y_{ni}-y_{nj} = y_{nij} \\
x_{ni}-y_{nj} = x_{nij} \\
\end{aligned}$$
Այսպիսով, $\Delta_{n}$ եռանկյան համար ստացանք հետևյալ հավասարումների համակարգը.
\begin{equation}
\scalemath{0.9}
{
\begin{cases}
2u_{nk}S_{n}\left(y_{nlm}^{2}+x_{nlm}^{2}\right)+2u_{nl}S_{n}\left(y_{nlm}y_{nmk}+x_{nlm}x_{nmk}\right)+2u_{nm}S_{n}\left(y_{nlm}y_{nkl}+x_{nlm}x_{nkl}\right)=-\dfrac{2}{3}u_{nk} S_{n}^{2}f_{nk} \\
u_{n} = 0, \text{եթե } u_{n}\text{-ն տիրույթի եզրի գագաթ է}
\end{cases}
}
\end{equation}
Համախմբելով բոլոր եռանկյունների համար ստացված հավասարումները, կստանանք գլխավոր համակարգը, և յուրաքանչյուր $n$-րդ գագաթի համար կունենանք.
\begin{equation}
\scalemath{0.8}
{
\begin{cases}
\sum_{r: P_{n} \in \Delta{r}}u_{nk}S_{r}\left(y_{rlm}^{2}+x_{rlm}^{2}\right)+u_{nl}S_{r}\left(y_{rlm}y_{rmk}+x_{rlm}x_{rmk}\right)+u_{nm}S_{r}\left(y_{rlm}y_{rkl}+x_{rlm}x_{rkl}\right)=-\dfrac{1}{3}\sum_{r: P_{n} \in \Delta{r}}u_{nk} S_{r}^{2}f_{nk} \\
u_{nk} = 0, \text{եթե } u_{nk}\text{-ն տիրույթի եզրի գագաթ է}
\end{cases}
}
\end{equation}

\subsubsection*{{\addfontfeatures{FakeBold=2.0}Ծրագրային իրականացում}}

Ինչպես ուղղանկյուն տիրույթի դեպքում, այնպես էլ այս դեպքում կօգտվենք նույն գործիքներից։ Տիրույթի եռանկյունացման համար կօգտվենք Python ծրագրավորման լեզվի triangle գրադարանից:

Ծրագրի սկզբում տրվում է տիրույթի եզրագծի կետերը, եզրագծերի կետերի միացման հաջորդականությունը, եռանկյունացման պարամետրերը և $f$ ֆունկցիան։ Հաջորդիվ յուրաքանչյուր եռանկյան համար կազմվում է հավասարումների համակարգը, հընթացս կառուցելով գլխավոր հավասարումների համակարգը։ Այնուհետև կանչվում է այն լուծող ֆունկցիան։ Այնուհետև բազիսային ֆունկցիաների միջոցով կառուցվում է մոտարկող ֆունկցիան։

Օրինակ

$$
  \begin{cases}
        \Delta u =2 \\
        u \Big |_{\partial D} = 0
  \end{cases}
$$

$$ D = \left\{\left(x,y\right): x^{2}+y^{2} \leq 1\right\}$$
Տիրույթի եռանկյունացումը
\begin{figure}[H]
\centering
\includegraphics[width=0.6\textwidth]{images/circle_mesh}
\captionsetup{labelformat=empty}
\caption{Նկար 5.3. Շրջանաձև տիրույթի եռանկյունացման գրաֆիկական ներկայացում։}
\end{figure}
Խնդրի համար ստացված լուծումը.
\begin{figure}[H]
\centering
\includegraphics[width=0.8\textwidth]{images/poisson_equation_on_unit_circle_solution}
\captionsetup{labelformat=empty}
\caption{Նկար 5.4. Պուասոնի հավասարման լուծման գրաֆիկական ներկայացում։}
\end{figure}
\newpage
\subsection*{{\addfontfeatures{FakeBold=2.0}5.3 Բիհարմոնիկ հավասարման լուծում ուղղանկյուն տիրույթում}}
Դիտարկենք հետևյալ դիֆերենցիալ հավասարումը.
\begin{equation}
\begin{dcases}
&\Delta^{2} u =f \\
&u \Big |_{\partial D} = 0\\
&\dfrac{\partial u}{\partial n} \Big |_{\partial D} = 0
\end{dcases}
\end{equation}
որտեղ $D = \left[x_{0}, x_{N}\right] \times \left[y_{0}, y_{M}\right]$:
Այս դիֆերենցիալ հավասարման  համապատասխան վարիացիոն խնդիրը կլինի.
\begin{equation}
I(u) = \frac{1}{2}\iint \limits_{D} \left[u_{xx}^{2} + 2u_{xy}^{2} + u_{yy}^{2} \right]dxdy - \iint \limits_{D} fudxdy \longrightarrow min
\end{equation}
$D$ տիրույթը տրոհենք ուղղանկյուների $h_{1}$ և $h_{2}$ քայլերով համապատասխանաբար ըստ $x$ և $y$ կոորդինատների։
Որոնելի ֆունկցիան կփնտրենք $\left(2.13\right)$ ում ներկայացված բազիսային ֆունկցիաների գծային կոմբինացիայի տեսքով.
\begin{equation}
u(x,y) = \sum_{i=0}^{N} \sum_{j=0}^{M} u_{ij}\varphi^{(ij)}(x,y)
\end{equation}
Հետևաբար ինտեգրալ ֆունկցիոնալը կունենք հետևյալ տեսքը.
\begin{equation}
$$\scalemath{0.7}{\frac{1}{2} \iint \limits_{D}\left[\left(\sum_{i=0}^{N} \sum_{j=0}^{M}u_{ij}\varphi^{(ij)}_{xx}(x,y)\right)^2 +  2\left(\sum_{i=0}^{N} \sum_{j=0}^{M}u_{ij}\varphi^{(ij)}_{xy}(x,y)\right)^2 + \left(\sum_{i=0}^{N} \sum_{j=0}^{M}u_{ij}\varphi^{(ij)}_{yy}(x,y)\right)^2-2f(x,y)\left(\sum_{i=0}^{N} \sum_{j=0}^{M}u_{ij}\varphi^{(ij)}(x,y)\right)\right]dxdy}
\end{equation}
Համաձայն էքստրեմումի անհրաժեշտ պայմանի.
\begin{equation}
\dfrac{\partial I}{ \partial u_{kl}} = \dfrac{\partial}{\partial u_{kl}} I \left(\sum_{i=0}^{N}\sum_{j=0}^{M} u_{ij}\varphi^{(ij)}(x,y)\right) = 0
\end{equation}
Դիտարկենք $\left(29\right)$-ի առաջին կրկնակի գումարը.
$$\left(\sum_{i=0}^{N} \sum_{j=0}^{M}u_{ij}\varphi_{xx}^{(ij)}(x,y)\right)^2 = \left[u_{kl}\varphi_{xx}^{(kl)}(x,y)\right]^{2} + 2u_{kl}\varphi_{xx}^{(kl)}(x,y)[\dots] + [\dots]^2$$
\noindent որտեղ բազմակետերով արտահայտությունը իր մեջ չի պարունակում $u_{kl}$ ը։
Հանգունորեն երկրորդ կրկնակի գումարի համար.
$$\left(\sum_{i=0}^{N} \sum_{j=0}^{M}u_{ij}\varphi_{xy}^{(ij)}(x,y)\right)^2 = \left[u_{kl}\varphi_{xy}^{(kl)}(x,y)\right]^{2} + 2u_{kl}\varphi_{xy}^{(kl)}(x,y)[\dots] + [\dots]^2$$
Հանգունորեն երրորդ կրկնակի գումարի համար.
$$\left(\sum_{i=0}^{N} \sum_{j=0}^{M}u_{ij}\varphi_{yy}^{(ij)}(x,y)\right)^2 = \left[u_{kl}\varphi_{yy}^{(kl)}(x,y)\right]^{2} + 2u_{kl}\varphi_{yy}^{(kl)}(x,y)[\dots] + [\dots]^2$$
Այսպիսով ավելի պարզեցված տեսքով համակարգը ունի հետևյալ տեսքը.
\begin{equation}\scalemath{0.65}{2\iint \limits_{D} \left[u_{kl}\left\{ \varphi_{xx}^{kl}(x,y)\right\}^{2} + \varphi_{xx}^{kl}(x,y)[\dots] + 2u_{kl}\left\{ \varphi_{xy}^{kl}(x,y)\right\}^{2} + 2\varphi_{xy}^{kl}(x,y)[\dots] +\left\{ \varphi_{yy}^{kl}(x,y)\right\}^{2} +u_{kl}\varphi_{yy}^{kl}(x,y)[\dots] +f(x,y)\varphi_{kl}(x,y)\right]dxdy=0}\end{equation}
Քանի որ տրված են Դիրիխլեի և Նեյմանի եզրային պայմանները, կդիտարկենք միայն $2 \leq k \leq M-2, 2 \leq l \leq N-2$ դեպքերը։
$\left(19\right)$ ինտեգրալը հաշվենք անդամ առ անդամ։
\begin{equation}
\iint \limits_{D} 2u_{kl}\left\{ \varphi_{xx}^{(kl)}(x,y)\right\}^{2}dxdy=2u_{kl}\int \limits_{x_{k-2}}^{x_{k+2}}\int \limits_{y_{l-2}}^{y_{l+2}} \left\{ B_{k}^{''}(x)B_{l}(y)\right\}^{2}dxdy=2u_{kl}\cdot \dfrac{6}{h_{1}^{3}} \cdot \dfrac{151}{140}h_{2}
\end{equation}
\begin{equation}\iint \limits_{D} 4u_{kl}\left\{ \varphi_{xy}^{(kl)}(x,y)\right\}^{2}dxdy=4u_{kl}\int \limits_{x_{k-2}}^{x_{k+2}}\int \limits_{y_{l-2}}^{y_{l+2}} \left\{ B_{k}^{'}(x)B_{l}^{'}(y)\right\}^{2}dxdy=4u_{kl}\cdot \dfrac{3}{2h_{1}} \cdot \dfrac{3}{2h_{2}}
\end{equation}
\begin{equation}
\iint \limits_{D} 2u_{kl}\left\{ \varphi_{yy}^{(kl)}(x,y)\right\}^{2}dxdy=2u_{kl}\int \limits_{x_{k-2}}^{x_{k+2}}\int \limits_{y_{l-2}}^{y_{l+2}} \left\{ B_{k}(x)B_{l}^{''}(y)\right\}^{2}dxdy=2u_{kl}\cdot \dfrac{151}{140}h_{2} \cdot \dfrac{6}{h_{1}^{3}}
\end{equation}
\begin{equation}
\begin{split}
\iint \limits_{D} 2f(x,y) \varphi^{(kl)}(x,y)dxdy = 2 \int \limits_{x_{k-2}}^{x_{k+2}}\int \limits_{y_{l-2}}^{y_{l+2}}f(x,y)B_{k}(x)B_{l}(y)dxdy\approx \\
\approx \dfrac{2}{9} h_{1}h_{2} \left[f_{k-1,l}+2f_{k,l}+f_{k+1,l}\right] \left[f_{k,l-1}+2f_{k,l}+f_{k,l+1}\right]
\end{split}
\end{equation}
Դիտարկենք մյուս ինտեգրալները։ Քանի որ $\varphi^{(ij)}$ բազիսային ֆունկցիաները ունեն լոկալ կրիչներ $4 \times 4 = 16$ ուղղանկյուն էլեմենտներում, ապա տրված $\varphi^{kl}$ ֆունկցիան հատվում է 49 այլ բազիսային ֆունկցիաների հետ։ Բացառություն են կազմում $i=2, M-2$, $j=2, N-2$ դեպքերը, որոնց համար լոկալ կրիչը $6 \times 7 = 42$ է եզրին հարող հանգույցների համար, և $6 \times 6 = 36$ անկյուններին հարող հանգույցների համար։ Ստորև ներկայացվում է երեք հնարավոր դեքերը (արտապատկերման համար հանգույցները պատկերված են ներկված ուղղանկյունների տեսքով)։
\begin{figure}[H]
  \centering
  \begin{minipage}[b]{0.2\textwidth}
    \includegraphics[width=\textwidth]{images/two_dimensional_basis_intersection}
    %\captionsetup{labelformat=empty}
  \end{minipage}
  \hfill
  \begin{minipage}[b]{0.2\textwidth}
    \includegraphics[width=\textwidth]{images/two_dimensional_basis_intersection_edge_1}
    %\captionsetup{labelformat=empty}
  \end{minipage}
\hfill
  \begin{minipage}[b]{0.2\textwidth}
    \includegraphics[width=\textwidth]{images/two_dimensional_basis_intersection_edge_2}
    %\captionsetup{labelformat=empty}
  \end{minipage}
\captionsetup{labelformat=empty}
\caption{Նկար 5.5. Բազիսային ֆունկցիաների հատումների ներկայացում տիրույթի ներսում, անկյունի վրա և եզրի վրա։}
\end{figure}
		$$\iint \limits_{D} 2\varphi_{xx}^{(kl)}(x,y)[\dots]dxdy, \iint \limits_{D} 4\varphi_{xy}^{(kl)}(x,y)[\dots]dxdy, \iint \limits_{D} 2\varphi_{yy}^{(kl)}(x,y)[\dots]dxdy$$
ինտեգրալները ներկայացնենք կրկնակի ինտեգրալների տեսքով։
\begin{equation}
\iint \limits_{D} 2\varphi_{xx}^{(kl)}(x,y)[\dots]dxdy= 2\sum_{\substack{i=k-3\\ i \neq k}}^{k+3}\sum_{\substack{j=l-3\\ j \neq l}}^{l+3} u_{ij}\int \limits_{x_{k-2}}^{x_{k+2}}B_{k}^{''}(x)B_{i}^{''}(x)dx \int \limits_{y_{l-2}}^{y_{l+2}}B_{l}(y)B_{j}(y)dy
\end{equation}
\begin{equation}
\iint \limits_{D} 4\varphi_{xy}^{(kl)}(x,y)[\dots]dxdy= 4\sum_{\substack{i=k-3\\ i \neq k}}^{k+3}\sum_{\substack{j=l-3\\ j \neq l}}^{l+3}u_{ij}\int \limits_{x_{k-2}}^{x_{k+2}}B_{k}^{'}(x)B_{i}^{'}(x)dx \int \limits_{y_{l-2}}^{y_{l+2}}B_{l}^{'}(y)B_{j}^{'}(y)dy
\end{equation}
\begin{equation}
\iint \limits_{D} 2\varphi_{yy}^{(kl)}(x,y)[\dots]dxdy= 2\sum_{\substack{i=k-3\\ i \neq k}}^{k+3}\sum_{\substack{j=l-3\\ j \neq l}}^{l+3}u_{ij}\int \limits_{x_{k-2}}^{x_{k+2}}B_{k}(x)B_{i}(x)dx \int \limits_{y_{l-2}}^{y_{l+2}}B_{l}^{''}(y)B_{j}^{''}(y)dy
\end{equation}
\newpage
Դիտարկենք  հետևյալ դեպքերը.
\begin{enumerate}
\item{Երբ $3 \leq k \leq M-3, \; 3 \leq l \leq N-3$ }

Առաջին ինտեգրալի համար
\begin{equation}
\begin{aligned}
&\int \limits_{x_{k-2}}^{x_{k+2}}B_{k}^{''}(x)B_{k-3}^{''}(x)dx = \dfrac{3}{8h_{1}^3}, \; \int \limits_{x_{k-2}}^{x_{k+2}}B_{k}^{''}(x)B_{k-2}^{''}(x)dx = 0, \; \int \limits_{x_{k-2}}^{x_{k+2}}B_{k}^{''}(x)B_{k-1}^{''}(x)dx = -\dfrac{27}{8h_{1}^3} \\
&\int \limits_{x_{k-2}}^{x_{k+2}}B_{k}^{''}(x)B_{k+1}^{''}(x)dx =-\dfrac{27}{8h_{1}^3}, \; \int \limits_{x_{k-2}}^{x_{k+2}}B_{k}^{''}(x)B_{k+2}^{''}(x)dx = 0, \; \int \limits_{x_{k-2}}^{x_{k+2}}B_{k}^{''}(x)B_{k+3}^{''}(x)dx =  \dfrac{3}{8h_{1}^3} \\
&\int \limits_{y_{l-2}}^{y_{l+2}}B_{l}(y)B_{l-3}(y)dy=\dfrac{h_{2}}{2240}, \; \int \limits_{y_{l-2}}^{y_{l+2}}B_{l}(y)B_{l-2}(y)dy=\dfrac{3h_{2}}{56}, \; \int \limits_{y_{l-2}}^{y_{l+2}}B_{l}(y)B_{l-1}(y)dy=\dfrac{1991h_{2}}{2240} \\
&\int \limits_{y_{l-2}}^{y_{l+2}}B_{l}(y)B_{l+1}(y)dy=\dfrac{1991h_{2}}{2240}, \; \int \limits_{y_{l-2}}^{y_{l+2}}B_{l}(y)B_{l+2}(y)dy=\dfrac{3h_{2}}{56}, \; \int \limits_{y_{l-2}}^{y_{l+2}}B_{l}(y)B_{l+3}(y)dy=\dfrac{h_{2}}{2240}
\end{aligned}
\end{equation}
Երկրորդ ինտեգրալի համար
\begin{equation}
\begin{aligned}
&\int \limits_{x_{k-2}}^{x_{k+2}}B_{k}^{'}(x)B_{k-3}^{'}(x)dx=\dfrac{-3}{160h_{1}}, \; \int \limits_{x_{k-2}}^{x_{k+2}}B_{k}^{'}(x)B_{k-2}^{'}(x)dx=\dfrac{-9}{20h_{1}}, \; \int \limits_{x_{k-2}}^{x_{k+2}}B_{k}^{'}(x)B_{k-1}^{'}(x)dx=\dfrac{-9}{32h_{1}} \\
&\int \limits_{x_{k-2}}^{x_{k+2}}B_{k}^{'}(x)B_{k+1}^{'}(x)dx=\dfrac{-9}{32h_{1}}, \; \int \limits_{x_{k-2}}^{x_{k+2}}B_{k}^{'}(x)B_{k+2}^{'}(x)dx=\dfrac{-9}{20h_{1}}, \; \int \limits_{x_{k-2}}^{x_{k+2}}B_{k}^{'}(x)B_{k+3}^{'}(x)dx=\dfrac{-3}{160h_{1}} \\
&\int \limits_{y_{l-2}}^{y_{l+2}}B_{l}^{'}(y)B_{l-3}^{'}(y)dy=\dfrac{-3}{160h_{2}}, \; \int \limits_{y_{l-2}}^{y_{l+2}}B_{l}^{'}(y)B_{l-2}^{'}(y)dy=\dfrac{-9}{20h_{2}}, \; \int \limits_{y_{l-2}}^{y_{l+2}}B_{l}^{'}(y)B_{l-1}^{'}(y)dy=\dfrac{-9}{32h_{2}} \\
&\int \limits_{y_{l-2}}^{y_{l+2}}B_{l}^{'}(y)B_{l+1}^{'}(y)dy=\dfrac{-9}{32h_{2}}, \; \int \limits_{y_{l-2}}^{y_{l+2}}B_{l}^{'}(y)B_{l+2}^{'}(y)dy=\dfrac{-9}{20h_{2}}, \; \int \limits_{y_{l+2}}^{y_{l+2}}B_{l}^{'}(y)B_{l+3}^{'}(y)dy=\dfrac{-3}{160h_{2}}
\end{aligned}
\end{equation}
Երրորդ ինտեգրալի համար
\begin{equation}
\begin{aligned}
&\int \limits_{x_{k-2}}^{x_{k+2}}B_{k}(x)B_{k-3}(x)dx=\dfrac{h_{1}}{2240}, \; \int \limits_{x_{k-2}}^{x_{k+2}}B_{k}(x)B_{k-2}(x)dx=\dfrac{3h_{1}}{56}, \; \int \limits_{x_{k-2}}^{x_{k+2}}B_{k}(x)B_{k-1}(x)dx=\dfrac{1991h_{1}}{2240} \\
&\int \limits_{x_{k-2}}^{x_{k+2}}B_{k}(x)B_{k+1}(x)dx=\dfrac{1991h_{1}}{2240}, \; \int \limits_{x_{k-2}}^{x_{l+2}}B_{k}(x)B_{k+2}(x)dx=\dfrac{3h_{1}}{56}, \; \int \limits_{x_{k-2}}^{x_{k+2}}B_{k}(x)B_{k+3}(x)dx=\dfrac{h_{1}}{2240} \\
&\int \limits_{y_{l-2}}^{y_{l+2}}B_{l}^{''}(y)B_{l-3}^{''}(y)dy = \dfrac{3}{8h_{2}^3}, \; \int \limits_{y_{l-2}}^{y_{l+2}}B_{l}^{''}(y)B_{l-2}^{''}(y)dy = 0, \; \int \limits_{y_{l-2}}^{y_{l+2}}B_{l}^{''}(y)B_{l-1}^{''}(y)dy = -\dfrac{27}{8h_{2}^3} \\
&\int \limits_{y_{l-2}}^{y_{l+2}}B_{l}^{''}(y)B_{l+1}^{''}(y)dy =-\dfrac{27}{8h_{2}^3}, \; \int \limits_{y_{l-2}}^{y_{l+2}}B_{l}^{''}(y)B_{l+2}^{''}(y)dy = 0, \; \int \limits_{y_{l-2}}^{y_{l+2}}B_{l}^{''}(y)B_{l+3}^{''}(y)dy =  \dfrac{3}{8h_{2}^3}
\end{aligned}
\end{equation}
\item{$k=2,  l =2 $}

Առաջին ինտեգրալի համար
\begin{equation}
\begin{aligned}
&\int \limits_{x_{0}}^{x_{4}}B_{2}^{''}(x)B_{0}^{''}(x)dx = \dfrac{3}{8h_{1}^3}, \; \int \limits_{x_{0}}^{x_{4}}B_{2}^{''}(x)B_{1}^{''}(x)dx = -\dfrac{27}{8h_{1}^3} \\
&\int \limits_{x_{0}}^{x_{4}}B_{2}^{''}(x)B_{3}^{''}(x)dx =-\dfrac{27}{8h_{1}^3}, \; \int \limits_{x_{0}}^{x_{4}}B_{2}^{''}(x)B_{4}^{''}(x)dx = 0, \; \int \limits_{x_{0}}^{x_{4}}B_{2}^{''}(x)B_{5}^{''}(x)dx =  \dfrac{3}{8h_{1}^3} \\
&\int \limits_{y_{0}}^{y_{4}}B_{2}(y)B_{0}(y)dy=\dfrac{121h_{2}}{2240}, \; \int \limits_{y_{0}}^{y_{4}}B_{2}(y)B_{1}(y)dy=\dfrac{1991h_{2}}{2240} \\
&\int \limits_{y_{0}}^{y_{4}}B_{2}(y)B_{3}(y)dy=\dfrac{1991h_{2}}{2240}, \; \int \limits_{y_{0}}^{y_{4}}B_{2}(y)B_{4}(y)dy=\dfrac{3h_{2}}{56}, \; \int \limits_{y_{0}}^{y_{4}}B_{2}(y)B_{5}(y)dy=\dfrac{h_{2}}{2240}
\end{aligned}
\end{equation}
Երկրորդ ինտեգրալի համար
\begin{equation}
\begin{aligned}
&\int \limits_{x_{0}}^{x_{4}}B_{2}^{'}(x)B_{0}^{'}(x)dx=-\dfrac{15}{32h_{1}}, \; \int \limits_{x_{0}}^{x_{4}}B_{2}^{'}(x)B_{1}^{'}(x)dx=-\dfrac{9}{32h_{1}} \\
&\int \limits_{x_{0}}^{x_{4}}B_{2}^{'}(x)B_{3}^{'}(x)dx=\dfrac{-9}{32h_{1}}, \; \int \limits_{x_{0}}^{x_{4}}B_{2}^{'}(x)B_{4}^{'}(x)dx=\dfrac{-9}{20h_{1}}, \; \int \limits_{x_{0}}^{x_{4}}B_{2}^{'}(x)B_{5}^{'}(x)dx=\dfrac{-3}{160h_{1}} \\
&\int \limits_{y_{0}}^{y_{4}}B_{2}^{'}(y)B_{0}^{'}(y)dy=-\dfrac{9}{32h_{2}}, \; \int \limits_{y_{0}}^{y_{4}}B_{2}^{'}(y)B_{1}^{'}(y)dy=-\dfrac{9}{32h_{2}} \\
&\int \limits_{y_{0}}^{y_{4}}B_{2}^{'}(y)B_{3}^{'}(y)dy=\dfrac{-9}{32h_{2}}, \; \int \limits_{y_{0}}^{y_{4}}B_{2}^{'}(y)B_{4}^{'}(y)dy=\dfrac{-9}{20h_{2}}, \; \int \limits_{y_{0}}^{y_{4}}B_{2}^{'}(y)B_{5}^{'}(y)dy=\dfrac{-3}{160h_{2}}
\end{aligned}
\end{equation}
Երրորդ ինտեգրալի համար
\begin{equation}
\begin{aligned}
&\int \limits_{x_{0}}^{x_{4}}B_{2}(x)B_{0}(x)dx=\dfrac{121h_{1}}{2240}, \; \int \limits_{x_{0}}^{x_{4}}B_{2}(x)B_{1}(x)dx=\dfrac{1991h_{1}}{2240} \\
&\int \limits_{x_{0}}^{x_{4}}B_{2}(x)B_{3}(x)dx=\dfrac{1991h_{1}}{2240}, \; \int \limits_{x_{0}}^{x_{4}}B_{2}(x)B_{4}(x)dx=\dfrac{3h_{1}}{56}, \; \int \limits_{x_{0}}^{x_{4}}B_{2}(x)B_{5}(x)dx=\dfrac{h_{1}}{2240} \\
&\int \limits_{y_{0}}^{y_{4}}B_{2}^{''}(y)B_{0}^{''}(y)dy = -\dfrac{3}{8h_{1}^{2}}, \; \int \limits_{y_{0}}^{y_{4}}B_{2}^{''}(y)B_{1}^{''}(y)dy = -\dfrac{27}{8h_{2}^3} \\
&\int \limits_{y_{0}}^{y_{4}}B_{2}^{''}(y)B_{3}^{''}(y)dy =-\dfrac{27}{8h_{2}^3}, \; \int \limits_{y_{0}}^{y_{4}}B_{2}^{''}(y)B_{4}^{''}(y)dy = 0, \; \int \limits_{y_{0}}^{y_{4}}B_{2}^{''}(y)B_{5}^{''}(y)dy =  \dfrac{3}{8h_{2}^3}
\end{aligned}
\end{equation}
\item{$k=M-2, l=N-2$}
Քանի որ բազիսային ֆունկցիաները սիմետրիկ են, ապա այդ դեպքում կատարվում են նույն հաշվարկները, ինչ նախորդ կետում։
\end{enumerate}
Հաշվի առնելով նախորդիվ հաշվարկված ինտեգրալները, ինչպես նաև Դիրիխլեի և Նեյմանի եզրային պայմանները, կստանանք հետևյալ հավասարումենրի համակարգը.
Դիրիխլեի եզրային պայմաններ.
\begin{equation}
\begin{dcases}
u_{00}+\frac{5}{16}(u_{01} + u_{10}) +\frac{1}{16}u_{11} &= 0\\
u_{M0}+\frac{5}{16}(u_{M1} + u_{M-1,0}) +\frac{1}{16}u_{M-1, 1} &= 0\\
u_{0N}+\frac{5}{16}(u_{1N} + u_{0,N-1}) +\frac{1}{16}u_{N-1, 1} &= 0\\
u_{MN}+\frac{5}{16}(u_{M-1,N} + u_{M,N-1}) +\frac{1}{16}u_{M-1,N-1} &= 0
\end{dcases}
\end{equation}
Նեյմանի եզրային պայմաններ.
\begin{equation}
\begin{dcases}
\dfrac{3}{4h_{2}}\left(u_{i1} - u_{i0}\right) + \dfrac{3}{16h_{2}}\left(u_{i-1, 0}+u_{i+1, 0}+u_{i-1, 1}+u_{i+1, 1}\right)&=0\\
\dfrac{3}{4h_{2}}\left(u_{iN} - u_{i, N-1}\right) + \dfrac{3}{16h_{2}}\left(u_{i-1, N-1}+u_{i+1, N-1}+u_{i-1, N-1}+u_{i+1, N-1}\right)&=0\\
\dfrac{3}{4h_{1}}\left(u_{1j} - u_{0j}\right) + \dfrac{3}{16h_{1}}\left(u_{0, j-1}+u_{0, j+1}+u_{1, j-1}+u_{1, j+1}\right)&=0\\
\dfrac{3}{4h_{1}}\left(u_{Mj} - u_{M-1,j}\right) + \dfrac{3}{16h_{1}}\left(u_{M-1,j-1}+u_{M-1, j+1}+u_{M-1, j-1}+u_{M-1, j+1}\right)&=0\\
\end{dcases}
\end{equation}
Մինիմումի պայմաններ.
\begin{equation}
\begin{gathered}
\scalemath{0.6}{
\left(
 \begin{bmatrix}
           \sfrac{3}{8h_{1}^{2}} \\
           0 \\
           -\sfrac{27}{8h_{1}^{3}} \\
	\sfrac{6}{h_{1}^{3}} \\
           -\sfrac{27}{8h_{1}^{3}} \\
           0 \\
           \sfrac{3}{8h_{1}^{2}} \\
\end{bmatrix} \otimes \begin{bmatrix}
           \sfrac{h_{2}}{2240} \\
           \sfrac{3h_{2}}{56} \\
           \sfrac{1991h_{2}}{2240} \\
	 \sfrac{151h_{2}}{140} \\
           \sfrac{1991h_{2}}{2240} \\
           \sfrac{3h_{2}}{56} \\
           \sfrac{h_{2}}{2240}\\
\end{bmatrix} 
+
 \begin{bmatrix}
           -\sfrac{3}{160h_{1}} \\
           -\sfrac{9}{20h_{1}} \\
	-\sfrac{9}{32h_{1}} \\
           \sfrac{3}{2h_{1}} \\
	-\sfrac{9}{32h_{1}} \\
           -\sfrac{9}{20h_{1}} \\
           -\sfrac{3}{160h_{1}} \\
\end{bmatrix} \otimes  \begin{bmatrix}
           -\sfrac{3}{160h_{1}} \\
           -\sfrac{9}{20h_{1}} \\
	-\sfrac{9}{32h_{1}} \\
           \sfrac{3}{2h_{1}} \\
	-\sfrac{9}{32h_{1}} \\
           -\sfrac{9}{20h_{1}} \\
           -\sfrac{3}{160h_{1}} \\
\end{bmatrix} 
+ 
 \begin{bmatrix}
           \sfrac{h_{2}}{2240} \\
           \sfrac{3h_{2}}{56} \\
           \sfrac{1991h_{2}}{2240} \\
	 \sfrac{151h_{2}}{140} \\
           \sfrac{1991h_{2}}{2240} \\
           \sfrac{3h_{2}}{56} \\
           \sfrac{h_{2}}{2240}\\
\end{bmatrix}\otimes \begin{bmatrix}
           \sfrac{3}{8h_{1}^{2}} \\
           0 \\
           -\sfrac{27}{8h_{1}^{3}} \\
	\sfrac{6}{h_{1}^{3}} \\
           -\sfrac{27}{8h_{1}^{3}} \\
           0 \\
           \sfrac{3}{8h_{1}^{2}} \\
\end{bmatrix}
\right)
\begin{bmatrix}
u_{k-3,l-3} & u_{k-2,l-3} & u_{k-1,l-3} & u_{k,l-3} & u_{k+1,l-3} & u_{k+2,l-3} & u_{k+3,l-3}\\
u_{k-3,l-2} & u_{k-2,l-2} & u_{k-1,l-2} & u_{k,l-2} & u_{k+1,l-2} & u_{k+2,l-2} & u_{k+3,l-2}\\
u_{k-3,l-1} & u_{k-2,l-1} & u_{k-1,l-1} & u_{k,l-1} & u_{k+1,l-1} & u_{k+2,l-1} & u_{k+3,l-1}\\
u_{k-3,l} & u_{k-2,l} & u_{k-1,l} & u_{k,l} & u_{k+1,l} & u_{k+2,l} & u_{k+3,l}\\
u_{k-3,l+1} & u_{k-2,l+1} & u_{k-1,l+1} & u_{k,l+1} & u_{k+1,l+1} & u_{k+2,l+1} & u_{k+3,l+1}\\
u_{k-3,l+2} & u_{k-2,l+2} & u_{k-1,l+2} & u_{k,l+2} & u_{k+1,l+2} & u_{k+2,l+2} & u_{k+3,l+2}\\
u_{k-3,l+3} & u_{k-2,l+3} & u_{k-1,l+3} & u_{k,l+3} & u_{k+1,l+3} & u_{k+2,l+3} & u_{k+3,l+3}\\
\end{bmatrix}}=\\ 
=\dfrac{2}{9} h_{1}h_{2} \left[f_{k-1,l}+2f_{k,l}+f_{k+1,l}\right] \left[f_{k,l-1}+2f_{k,l}+f_{k,l+1}\right] 
\end{gathered}
\end{equation}
$k=2, M-2$, ինչպես նաև $l=2, N-2$ -ի համար կազմվում է նույն հավասարումը, փոխելով միայն համապասխան գործակիցները և դրանց քանակը։
\newpage
\subsubsection*{{\addfontfeatures{FakeBold=2.0}Ծրագրային իրականացում}}
Ինչպես Պուասոնի հավասարման դեպքում, այնպես էլ այս դեպքում կօգտվենք նույն գործիքներից։
Որպես օրինակ լուծենք հետևյալ հավասարումը տրված կոնկրետ տիրույթով և $f$ ֆունկցիայով։
$$\begin{dcases}
								\Delta^{2} u &=1 \\
								u \Big |_{\partial D} &= 0\\
								\dfrac{\partial u}{\partial n} \Big |_{\partial D} &= 0
\end{dcases}$$
որտեղ $D = \left[0, 1\right] \times \left[-1, 1\right], \; h_{1}=0.01, h_{2}=0.02$։

Խնդրի համար ստացված լուծումը.
\begin{figure}[H]
\centering
\includegraphics[width=0.9\textwidth]{images/biharmonic_equation_solution}
\captionsetup{labelformat=empty}
\caption{Նկար 5.6. Բիհարմոնիկ հավասարման լուծման գրաֆիկական ներկայացում։}
\end{figure}

\newpage
\subsection*{{\addfontfeatures{FakeBold=2.0}5.4 Բիհարմոնիկ հավասարման լուծում եռանկյունացվող տիրույթում}}
Դիտարկենք հետևյալ դիֆերենցիալ հավասարումը.
\begin{equation}
\begin{dcases}
&\Delta^{2} u =f \\
&u \Big |_{\partial D} = 0\\
&\dfrac{\partial u}{\partial n} \Big |_{\partial D} = 0
\end{dcases}
\end{equation}
որտեղ $D$-ն կամայական կապակցված միակապ կոմպակտ տիրույթ է։
Այս դիֆերենցիալ հավասարման  համապատասխան վարիացիոն խնդիրը կլինի.
\begin{equation}
I(u) = \frac{1}{2}\iint \limits_{D} \left[u_{xx}^{2} + 2u_{xy}^{2} + u_{yy}^{2} \right]dxdy - \iint \limits_{D} fudxdy \longrightarrow min
\end{equation}
Ինչպես Պուասոնի հավասարման լուծման դեպքում, այստեղ ևս տիրույթը կտրոհենք եռանկյունների, սակայն քանի որ այս դեպքում պահանջվում է բազիսային ֆունկցիաների առնվազն $C^{1}$ կարգ, ուստի որպես բազիսային ֆունկցիաներ կվերցնենք Արգիրիսի բազիսային ֆունկցիաները։

Հաշվարկների և բանաձևերի ներկայացման հեշտության համար բազիսային ֆունկցիաները դասավորենք ըստ գագաթների հետևյալ հաջորդականությամբ.
\begin{equation}
\begin{aligned}
&\Psi_{(p, q)}, \; p=\overline{1, 3}, \; q = \overline{1, 7} \\
&\Psi^{(p, q)} = \left[\varphi_{p}^{(0, 0)}, \varphi_{p}^{(1, 0)}, \varphi_{p}^{(0, 1)}, \varphi_{p}^{(2, 0)}, \varphi_{p}^{(1, 1)}, \varphi_{p}^{(0, 2)}, \hat{\varphi}_{p}\right]
\end{aligned}
\end{equation}
Յուրաքանչյուր $\Delta_{n}$ եռանկյան վրա ($n=\overline{1, N}$)  որոնելի ֆունկցիան փնտրենք բազիսային ֆունկցիաների գծային կոմբինացիայի տեսքով.
\begin{equation}
u(x, y) = \sum_{i=1}^{3}\sum_{j=1}^{7}u_{nij}\Psi^{(ni,j)}\left(x,y\right)
\end{equation}
որտեղ $nij$-ն $n$-րդ եռանկյան $i$-րդ գագաթի $j$-րդ ֆունկցիայի համապատասխան ինդեքսն է, իսկ $\Psi^{(ni, j)}(x,y)$-ն այդ գագաթի համապատասխան բազիսային ֆունկցիան է։

Համաձայն էքստրեմումի անհրաժեշտ պայմանի.
\begin{equation}
\dfrac{\partial I}{ \partial u_{nkl}} = \dfrac{\partial}{\partial u_{nkl}} I \left(\sum_{i=1}^{3}\sum_{j=1}^{7}u_{nij}\Psi^{(ni,j)}\right) = 0
\end{equation}

$(5.51)$-ը հերթով տեղադրենք $(5.49)$-ի $u_{xx}^{2}, \; u_{xy}^{2}, \; u_{yy}^{2}, \; \text{և} \; f u$-ի մեջ և հաշվենք $(5.52)$-ը։
\begin{equation}
\begin{aligned}
&\dfrac{\partial}{ \partial u_{nkl}}u_{xx}^2 = \dfrac{\partial}{ \partial u_{nkl}}\left[\sum_{i=1}^{3}\sum_{j=1}^{7}u_{nij}\Psi^{(ni,j)}_{xx}\right]^{2} = 2\Psi^{(ni,j)}_{xx}\left[\sum_{i=1}^{3}\sum_{j=1}^{7}u_{nij}\Psi^{(ni,j)}_{xx}\right] \\
&\dfrac{\partial}{ \partial u_{nkl}}u_{xy}^2 = \dfrac{\partial}{ \partial u_{nkl}}\left[\sum_{i=1}^{3}\sum_{j=1}^{7}u_{nij}\Psi^{(ni,j)}_{xy}\right]^{2} = 2\Psi^{(ni,j)}_{xy}\left[\sum_{i=1}^{3}\sum_{j=1}^{7}u_{nij}\Psi^{(ni,j)}_{xy}\right] \\
&\dfrac{\partial}{ \partial u_{nkl}}u_{yy}^2 = \dfrac{\partial}{ \partial u_{nkl}}\left[\sum_{i=1}^{3}\sum_{j=1}^{7}u_{nij}\Psi^{(ni,j)}_{yy}\right]^{2} = 2\Psi^{(ni,j)}_{yy}\left[\sum_{i=1}^{3}\sum_{j=1}^{7}u_{nij}\Psi^{(ni,j)}_{yy}\right] \\
\end{aligned}
\end{equation}
Աջ մասի $f$ ֆունկցիան ևս ներկայացնենք բազիսային ֆունկցիաների գծային կամբինացիաների տեսքով:
\begin{equation}
\dfrac{\partial}{ \partial u_{nkl}}fu = \Psi^{(nk,l)}\left[\sum_{i=1}^{3}\sum_{j=1}^{7}f_{nij}\Psi^{(ni,j)}\right]
\end{equation}
Հիմնվելով $(5.53)$ և $(5.54)$-ի վրա, կունենանք.
\begin{equation}
\begin{gathered}
\dfrac{\partial I}{ \partial u_{nkl}} = \frac{1}{2}\iint \limits_{\Delta_{n}} \left[\dfrac{\partial}{ \partial u_{nkl}}u_{xx}^2 + 2\dfrac{\partial}{ \partial u_{nkl}}u_{xy}^2 + \dfrac{\partial}{ \partial u_{nkl}}u_{yy}^2 \right]dxdy - \iint \limits_{\Delta_{n}} \dfrac{\partial}{ \partial u_{nkl}}fudxdy = \\
\scalemath{0.8}{
\sum_{i=1}^{3}\sum_{j=1}^{7}\left\{u_{nij}\iint \limits_{\Delta_{n}}\left[\Psi^{(ni,j)}_{xx}\Psi^{(nk,l)}_{xx} + 2\Psi^{(ni,j)}_{xy}\Psi^{(nk,l)}_{xy} + \Psi^{(ni,j)}_{yy}\Psi^{(nk,l)}_{yy}\right]dxdy - f_{nij}\iint \limits_{\Delta_{n}} \left[\Psi^{(nk,l)}\Psi^{(ni,j)}\right]dxdy\right\}}=0
\end{gathered}
\end{equation}

Հաշվենք $(5.55)$-ի առաջին և երկրորդ ինտեգրալները։

Նշանակենք
\begin{equation}
\begin{aligned}
&\hat{A}_{ij} = \iint \limits_{\Delta_{n}} F\left(\hat{\Psi}^{(ni,j)}_{xx}\right)F\left(\hat{\Psi}^{(nk,l)}_{xx}\right) + 2F\left(\hat{\Psi}^{(ni,j)}_{xy}\right)F\left(\hat{\Psi}^{(nk,l)}_{xy}\right) + F\left(\hat{\Psi}^{(ni,j)}_{yy}\right)F\left(\hat{\Psi}^{(nk,l)}_{yy}\right)dxdy \\
&\hat{B}_{ij} = \iint \limits_{\Delta_{n}} F\left(\hat{\Psi}^{(ni,j)}\right)F\left(\hat{\Psi}^{(nk,l)}\right)dxdy
\end{aligned}
\end{equation}
 Ինտեգրալների համար կունենանք.
\begin{equation}
\begin{aligned}
& A_{ij} = M\hat{A}_{ij}M^{T} \\
& B_{ij} = M\hat{B}_{ij}M^{T}
\end{aligned}
\end{equation}
որտեղ $F$-ը $(2.41)$-ում սահմանված աֆինյան ձևափոխությունն է, իսկ $M$-ը՝ տրանսֆորմացիան։

Համախմբելով բոլոր եռանկյունների համար ստացված հավասարումները, կստանանք գլխավոր համակարգը, և յուրաքանչյուր $n$-րդ գագաթի համար կունենանք.
\begin{equation}
\begin{cases}
\sum_{r: P_{n}\in \Delta{r}}\sum_{j=1}^{7}  A_{nrj} = \sum_{r: P_{n}\in \Delta{r}}\sum_{j=1}^{7}B_{nrj}\\
u_{n1} = 0, \text{եթե } u_{n}\text{-ն տիրույթի եզրի գագաթ է} \\
u_{n7} = 0, \text{եթե } u_{n}\text{-ին համապատասխանող միջնակետը տիրույթի եզրի գագաթ է}
\end{cases}
\end{equation}

\newpage
\subsubsection*{{\addfontfeatures{FakeBold=2.0}Ծրագրային իրականացում}}

Ինչպես նախորդ դեպքերում, այնպես էլ այս դեպքում կօգտվենք նույն գործիքներից։
Հաշվարկները հեշտացնելու համար օգտագործվենք է sympy գրադարանը, որը թույլ է տալիս կատարել սիմվոլիկ հաշվարկներ, ավելի կոնկրետ, դիֆերենցում և ինտեգրում։ Նախորոք հաշվարկվել և ֆունկցիաների տեսքով պահպանվել են $(5.56)$ կետում նշված ինտեգրալները։ Հետագայում ծրագրի ընթացքում այդ ֆունկցիաներին տրվում են $(5.57)$ բանաձևի անհրաժեշտ փոփոխականները։ Այսպիսով, լուծումն ավելի ճշգրիտ է ստացվում, ի հաշիվ այն բանի, որ չեն օգտագործվում ինտեգրալների արժեքները մոտարկող քառակուսացման բանաձևեր։

Ծրագրի սկզբում տրվում է տիրույթի եզրագծի կետերը, եզրագծերի կետերի միացման հաջորդականությունը, եռանկյունացման պարամետրերը և $f$ ֆունկցիան։ Հաջորդիվ յուրաքանչյուր եռանկյան համար կազմվում է հավասարումների համակարգը, հընթացս կառուցելով գլխավոր հավասարումների համակարգը։ Այնուհետև կանչվում է այն լուծող ֆունկցիան։ Այնուհետև բազիսային ֆունկցիաների միջոցով կառուցվում է մոտարկող ֆունկցիան։

Օրինակ

$$
\begin{dcases}
&\Delta^{2} u = 1 \\
&u \Big |_{\partial D} = 0\\
&\dfrac{\partial u}{\partial n} \Big |_{\partial D} = 0
\end{dcases}
$$
$$ D = \left\{\left(x,y\right): x^{2}+y^{2} \leq 1\right\}$$
Տիրույթի եռանկյունացումը
\begin{figure}[H]
\centering
\includegraphics[width=0.3\textwidth]{images/circle_mesh_quad_biharmonic}
\captionsetup{labelformat=empty}
\caption{Նկար 5.7. Շրջանաձև տիրույթի եռանկյունացման գրաֆիկական ներկայացում։}
\end{figure}
\newpage
Խնդրի համար ստացված լուծումը.
\begin{figure}[H]
\centering
\includegraphics[width=0.7\textwidth]{images/biharmonic_equation_solution_argyris_function_value}
\captionsetup{labelformat=empty}
\caption{Նկար 5.8. Պուասոնի հավասարման լուծման գրաֆիկական ներկայացում։}
\end{figure}

\end{document}